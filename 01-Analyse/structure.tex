\newcommand{\TikzGraph}[4]
{
\draw [very thin, gray] (#1,#2) grid (#3,#4);
\draw[thick,->, gray] (#1,0) -- (#3,0);
\draw (#3,0) node[right] {$x$};
\draw [thick,->, gray] (0,#2) -- (0,#4);
\draw (0,#4) node[above] {$y$};
}

\newcommand{\TikzGraphC}[4]
{
\draw [very thin, gray!20] (#1,#2) grid (#3,#4);
\draw [thick,->, gray] (#1,0) -- (#3,0);
\draw (#3,0) node[right] {$\Re$};
\draw [thick,->, gray] (0,#2) -- (0,#4);
\draw (0,#4) node[above] {$\Im$};
}

\newcommand{\TikzFunc}[4]
{
\draw [samples=#4,domain=#1:#2] plot(\x,{#3});
}

%\begin{center}
%\begin{tikzpicture}[scale=0.5]
%\TikzGraph{-4}{-1}{4}{6}
%\TikzFunc{-3}{1.8}{exp(\x)}{25}
%\TikzFunc{-pi}{pi}{sin(\x r)}{25}
%\end{tikzpicture}
%\end{center}

\newcommand{\CirPole}[3]
{
\draw (#1,#2) node {$\bullet$};
\draw [thick,red] (#1,#2) circle (#3);
\draw [thick,->](#1-0.01,#2-#3) -- (#1,#2-#3) ;
}
\newcommand{\CirPoleB}[3]
{
\draw (#1,#2) node {$\bullet$};
\draw [thick,red] (#1,#2) circle (#3);
\draw [thick,->](#1+0.01,#2-#3) -- (#1,#2-#3) ;
}

\newcommand{\TikzPathA}[5]
{
\begin{tikzpicture}[scale=1]
\draw (0,0) node {$.$} ;
\draw (0,0) node[above] {#4};
\draw (0,-#2) node[below] {#5};
\draw [thick,-] (-#2,#1) -- (-#2,-#1);
\draw [thick,-] (-#3,#1) -- (-#3,-#1);
\draw [thick,-] (#2,#1) -- (#3,#1);
\draw [thick,-] (#3,-#1) -- (#2,-#1);
\draw [thick,<-,>=stealth] (#2,#1) arc (0:180:#2);
\draw [thick,->,>=stealth] (#3,#1) arc (0:180:#3);
\draw [thick,<-,>=stealth] (#3,-#1) arc (0:-180:#3);
\draw [thick,->,>=stealth] (#2,-#1) arc (0:-180:#2);
\end{tikzpicture}
}

\newcommand{\TikzPathAn}[5]
{
\begin{tikzpicture}[scale=1]
\draw (0,0) node {$.$} ;
\draw (0,0) node[above] {#4};
\draw (0,-#2) node[below] {#5};
\draw [thick,-] (-#2,#1) -- (-#2,-#1);
\draw [thick,-] (-#3,#1) -- (-#3,-#1);
\draw [thick,-] (#2,#1) -- (#3,#1);
\draw [thick,-] (#3,-#1) -- (#2,-#1);
\draw [thick,->,>=stealth] (#2,#1) arc (0:180:#2);
\draw [thick,<-,>=stealth] (#3,#1) arc (0:180:#3);
\draw [thick,->,>=stealth] (#3,-#1) arc (0:-180:#3);
\draw [thick,<-,>=stealth] (#2,-#1) arc (0:-180:#2);
\end{tikzpicture}
}

\newcommand{\TikzPathB}[4]
{
\begin{tikzpicture}[scale=1]
\draw (0,0) node {$.$} ;
\draw (0,0) node[above] {#3};
\draw (0,-#2) node[below] {#4};
\draw [thick,-] (-#2,#1) -- (-#2,-#1);
\draw [thick,-] (#2,#1) -- (#2,-#1);
\draw [thick,<-,>=stealth] (#2,#1) arc (0:180:#2);
\draw [thick,->,>=stealth] (#2,-#1) arc (0:-180:#2);
\end{tikzpicture}
}

\newcommand{\TikzPathBn}[4]
{
\begin{tikzpicture}[scale=1]
\draw (0,0) node {$.$} ;
\draw (0,0) node[above] {#3};
\draw (0,-#2) node[below] {#4};
\draw [thick,-] (-#2,#1) -- (-#2,-#1);
\draw [thick,-] (#2,#1) -- (#2,-#1);
\draw [thick,->,>=stealth] (#2,#1) arc (0:180:#2);
\draw [thick,<-,>=stealth] (#2,-#1) arc (0:-180:#2);
\end{tikzpicture}
}

\newcommand{\TikzPathC}[9]
{
\draw (0,0) node {$\bullet$} ;
\draw (0,0) node[below] {#3};
\draw [color=red](0,#1) node[below] {#4};
\draw [color=red](0,#2) node[above] {#5};
\draw [color=red,thick,<-,>=stealth] (#1,0) -- (#2,0) node[above,midway] {#6};
\draw [color=red,thick,<-,>=stealth] (-#2,0) -- (-#1,0) node[above,midway] {#7};
\draw [color=red,thick,->,>=stealth] (#1,0) arc (0:180:#1);
\draw [color=red,thick,<-,>=stealth] (#2,0) arc (0:180:#2);
\draw (#1,0) node[below] {#8};
\draw (#2,0) node[below] {#9};
\draw (-#1,0) node[below] {-#8};
\draw (-#2,0) node[below] {-#9};
}

\newcommand{\TikzPathD}[2]
{
\draw [color=red][thick,->,>=stealth] (0,0) -- (#1,0) ;
\draw [color=red][thick,->,>=stealth] (-#1,0) -- (0,0) ;
\draw [color=red][thick,->,>=stealth] (#1,0) arc (0:180:#1);
\draw (#1,0) node[below] {#2};
\draw (-#1,0) node[below] {-#2};
}

\newcommand{\TikzFrise}[4]
{
\begin{tikzpicture}[scale=1]
\draw [thick,-] (-#2,0) -- (#2,0);
\draw [thick,-] (-#2,0) -- (-#2,#1);
\draw [thick,-] (#2,0) -- (#2,#1);
\draw [thick,-] (#2,#1) -- (#2+1,#1);
\draw [thick,-] (-#2,#1) -- (-#2-1,#1);
\draw [thick,-] (#2+1,#1) -- (0,#1+3);
\draw [thick,-] (-#2-1,#1) -- (0,#1+3);
\foreach \k in {1,2,...,#1}
{
\pgfmathsetmacro\d{int(#3+20*\k)}
\draw [thick,-] (-#2,\k) -- (-#2+0.3,\k) node[right] {\d};
}
\end{tikzpicture}
}


%///////////////////////////////////////////////////////////
%STYLE & MISE EN PAGE

\geometry{hmargin=1.5cm,vmargin=1.5cm}
\setlength{\parindent}{0cm}

\bibliographystyle{apalike-fr}

\newcommand{\mycolor}{MidnightBlue}
\sectionfont{\color{\mycolor}}
\subsectionfont{\color{\mycolor}}
\subsubsectionfont{\color{\mycolor}}
\hypersetup{
    colorlinks=true,
    linkcolor=black,
    citecolor=\mycolor
}
\newcommand{\graphsize}{0.3}
\newcommand{\proof}{\color{Orange}{(d�monstration requise)}}
\newcommand{\here}{\color{Blue}{(<---ici)}}
\newcommand{\workon}{\color{Blue}{(en cours)}}
\newcommand{\verif}{\color{Orange}{(� v�rifier)}}
\newcommand{\todo}{\color{PineGreen}{(� faire)}}
\newcommand{\fig}{\color{PineGreen}{(figure requise)}}

\pagestyle{fancy}
\fancyhf{} 
\rfoot{Page \thepage \hspace{1pt} of \pageref{LastPage}}

%///////////////////////////////////////////////////////////
%//MATHS

\newcommand{\Heav}{\mathcal{H}}
\newcommand{\VP}{ \mathcal{V}.\mathcal{P}. }

%//CEIL FLOOR
\providecommand{\floor}[1]{\left \lfloor #1 \right \rfloor }
\providecommand{\ceil}[1]{\left \lceil #1 \right \rceil }

%//OPERATEURS
\newcommand{\op}[1]{\mathbf{#1}}
\newcommand{\opp}[2]{\op{#1}^{#2}}
\newcommand{\applyop}[2]{\op{#1}\bm{\left[}#2\bm{\right]}}
\newcommand{\applyopp}[3]{\op{#1}^{#2}\bm{\left[}#3\bm{\right]}}

%//PRODUIT SCALAIRE
\newcommand{\scal}[2]{\langle #1 \mid #2\rangle}

%//INTEGRATION PAR PARTIE
\newcommand{\intp}[6]
{\left\{
\begin{array}{ll}
\int (uv')=[uv]-\int (u'v)&\\
u=#1 & u'=#2\\
v'=#3 & v=#4\\
%\int (#1 #3)=[#1 #4]_{#5}^{#6}-\int_{#5}^{#6} #2 #4&\\
\end{array}
\right.}

%///////////////////////////////////////////////////////////
%//FIGURES

\newcommand{\graphtriplefunction}[6]
{
\begin{figure}[H]
\centering
\begin{tabular}{ccc}
\includegraphics[scale=0.3]{./img/cpp/#1.png}&
\includegraphics[scale=0.3]{./img/cpp/#3.png}&
\includegraphics[scale=0.3]{./img/cpp/#5.png}\\
\end{tabular}
\caption{Graphique des fonctions #2 et #4 et #6.}
\end{figure}
}
\newcommand{\graphdoublefunction}[4]
{
\begin{figure}[H]
\centering
\begin{tabular}{cc}
\includegraphics[scale=0.4]{./img/cpp/#1.png}&
\includegraphics[scale=0.4]{./img/cpp/#3.png}\\
\end{tabular}
\caption{Graphique des fonctions #2 et #4.}
\end{figure}
}
\newcommand{\graphfunction}[2]
{
\begin{figure}[H]
\centering
\includegraphics[scale=0.35]{./img/cpp/#1.png}
\caption{Graphique de la fonction #2.}
\end{figure}
}
\newcommand{\graphpartiallow}[3]
{
Si on regarde la convergence des sommes partielles dans le plan complexe on obtient:
\begin{figure}[H]
\centering
\begin{tabular}{cccccc}
$n=2$&
$n=4$&
$n=8$&
$n=16$&
$n=32$&
#2\\
\includegraphics[scale=0.15]{./img/cpp/#1#3_2.png}&
\includegraphics[scale=0.15]{./img/cpp/#1#3_4.png}&
\includegraphics[scale=0.15]{./img/cpp/#1#3_8.png}&
\includegraphics[scale=0.15]{./img/cpp/#1#3_16.png}&
\includegraphics[scale=0.15]{./img/cpp/#1#3_32.png}&
\includegraphics[scale=0.15]{./img/cpp/#1.png}\\
\end{tabular}
\caption{Convergence des sommes partielles de #2.}
\end{figure}
}
\newcommand{\graphpartialhigh}[3]
{
Si on regarde la convergence des sommes partielles dans le plan complexe on obtient:
\begin{figure}[H]
\centering
\begin{tabular}{cccccc}
$n=4$&
$n=8$&
$n=16$&
$n=32$&
$n=64$&
#2\\
\includegraphics[scale=0.15]{./img/cpp/#1#3_4.png}&
\includegraphics[scale=0.15]{./img/cpp/#1#3_8.png}&
\includegraphics[scale=0.15]{./img/cpp/#1#3_16.png}&
\includegraphics[scale=0.15]{./img/cpp/#1#3_32.png}&
\includegraphics[scale=0.15]{./img/cpp/#1#3_64.png}&
\includegraphics[scale=0.15]{./img/cpp/#1.png}\\
\end{tabular}
\caption{Convergence des sommes partielles de #2.}
\end{figure}
}

%///////////////////////////////////////////////////////////
%//MACROS & TITLE
\newcommand{\rinf}{\textbf{Rayon de convergence:} La fonction de poss�de pas de p�le le rayon de convergence est donc infini: $\boxed{r=\infty}$\\}
\newcommand{\notes}[1]{\textbf{\underline{Notes:}}\textit{ "#1"}}