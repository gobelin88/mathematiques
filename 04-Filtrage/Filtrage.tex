%\documentclass[draft]{article}
\documentclass{article}

\usepackage{color}
\usepackage[usenames,dvipsnames]{xcolor}
\usepackage[latin1]{inputenc}
\usepackage{graphicx}
\usepackage{multicol}
\usepackage{amsmath}
\usepackage{geometry}
\usepackage{listings}
\usepackage{hyperref}
\usepackage{amsfonts}

\providecommand{\floor}[1]{\left \lfloor #1 \right \rfloor }
\providecommand{\ceil}[1]{\left \lceil #1 \right \rceil }

\geometry{hmargin=1.5cm,vmargin=1.5cm}

\hypersetup{
    colorlinks=true,
    linkcolor=black,
}

\newcommand{\workon}{\color{Blue}{(en cours)}}
\newcommand{\verif}{\color{Orange}{(� v�rifier)}}
\newcommand{\todo}{\color{PineGreen}{(� faire)}}

\title{\textbf{Conception de filtres num�riques}\\Recueil de m�thodes}
\author{Lo�c Huguel}

\setlength{\parindent}{0cm}

\begin{document}

\maketitle
\tableofcontents
\newpage
\section{Filtres � r�ponse impulsionnelle finie : RIF}

\subsection{D�finition}

On d�finit un filtre � r�ponse finie par ses coefficients $h_n$ avec $n\in[0:N-1]$, dans le domaine temporel, on obtient le signal filtr� $\bar{S}(k)$ par convolution :

\[
	\boxed{\bar{S}(k)=\sum_{n=0}^{N-1} h_n S(k-n)}
\]

Alors la r�ponse fr�quentielle du filtre peut �tre calcul�e comme:
\[
\boxed{A(w)=\sum_{n=0}^M a_n cos(w n)}
\]

\[
	a_0=h_M \quad a_n=2 h_{M-n} \quad et \quad M=\frac{N-1}{2}
\]

\subsection{Conception par les moindres carr�s}
On d�finit le crit�re d'erreur $\epsilon$ par :
\[
\epsilon=\int_0^\pi W(w)(A(w)-D(w))^2 dw
\]

Ou 

\begin{center}
\begin{tabular}{cl}
$W(w)$&Est une fonction de pond�ration non n�gative\\
$A(w)$&Est l'amplitude actuelle\\
$D(w)$&Est l'amplitude d�sir�e. C'est � dire le gabarit du filtre\\
\end{tabular}
\end{center}

On doit donc trouver les coefficients $a_n$ tel que $\epsilon$ soit minimal, c'est � dire que:

\[
	\frac{\partial \epsilon}{\partial  a_k}=0
\]

Calculons les d�riv�es:

\[
\frac{\partial \epsilon}{\partial a_k}=\int_0^\pi \frac{\partial \left( W(w)(A(w)-D(w))^2 \right)}{\partial a_k}  dw
\]

Comme $(g(u))'=u'g'(u)$ et que seul $A(w)$ d�pend de $a_k$:

\[
\frac{\partial \epsilon}{\partial a_k}=2\int_0^\pi \frac{\partial A(w)}{\partial a_k} W(w)(A(w)-D(w))   dw
\]

Et de plus $\frac{\partial A(w)}{\partial a_k}=cos(w k)$, donc finalement 

\[
\frac{\partial \epsilon}{\partial a_k}=2\int_0^\pi cos(w k) W(w)(A(w)-D(w))   dw
\]

Si on r�sout :

\[
2\int_0^\pi cos(w k) W(w)(A(w)-D(w))   dw=0
\]
\[
\int_0^\pi cos(w k) W(w) A(w)   dw=\int_0^\pi cos(w k) W(w) D(w)   dw
\]

Mais $A(w)=\sum_{n=0}^M a_n cos(w n)$, donc :

\[
\int_0^\pi cos(w k) W(w) \left(\sum_{n=0}^M a_n cos(w n)\right)   dw=\int_0^\pi cos(w k) W(w) D(w)   dw
\]

\[
\sum_{n=0}^M a_n \int_0^\pi  W(w) cos(w k) cos(w n)  dw=\int_0^\pi  W(w) D(w) cos(w k)  dw
\]

Si on d�finit 
\[
\boxed{Q(k,n)=\int_0^\pi  W(w) cos(w k) cos(w n)  dw}
\]
Et 
\[
\boxed{B(k)=\int_0^\pi  W(w) D(w) cos(w k)  dw}
\]
Et 
\[
\boxed{A(n)=a_n}
\]

Alors on peut �crire le syst�me lin�aire:
\[
Q \cdot A= B 
\]

On peut alors r�soudre par SVD par exemple: $A=Q^+B$, il nous reste � pr�ciser comment calculer les int�grales de $Q$ et $B$. Comme on � l'identit�:

\[
cos(w k) cos(w n) = \frac{cos(w (k-n))+cos(w (k+n))}{2}
\]

Il s'en suit que:
\[
Q(k,n)=\int_0^\pi  W(w) \frac{cos(w (k-n))+cos(w (k+n))}{2}  dw
\]
\[
Q(k,n)=\frac{1}{2}\int_0^\pi  W(w) cos(w (k-n)) dw + \frac{1}{2}\int_0^\pi  W(w) cos(w (k+n))  dw
\]

Finalement
\[
\boxed{\left\{
\begin{array}{l}
Q(k,n)=\frac{1}{2}Q_1(k,n)+\frac{1}{2}Q_2(k,n)\\
Q_1(k,n)=q(k-n)\\
Q_2(k,n)=q(k+n)\\
q(k)= \frac{1}{2}\int_0^\pi  W(w) cos(w k) dw
\end{array}
\right.}
\]


\end{document}