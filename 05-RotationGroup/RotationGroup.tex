

\documentclass[12pt,a4paper]{article}
\usepackage{geometry}
\usepackage{amsmath}
\usepackage{amsfonts}
\usepackage{graphicx}
\geometry{hmargin=1.0cm,vmargin=2.0cm}
\setlength{\parindent}{0cm}
\title{Le Groupe de Rotations}
\author{Loïc Huguel}

\usepackage{mathtools}
\DeclarePairedDelimiter\ceil{\lceil}{\rceil}
\DeclarePairedDelimiter\floor{\lfloor}{\rfloor}

\newcommand{\Jx}
{
\begin{pmatrix}
	0&0&0\\
	0&0&-1\\
	0&1&0\\
\end{pmatrix}
}

\newcommand{\Jy}
{
	\begin{pmatrix}
		0&0&1\\
		0&0&0\\
		-1&0&0\\
	\end{pmatrix}
}

\newcommand{\Jz}
{
	\begin{pmatrix}
		0&-1&0\\
		1&0&0\\
		0&0&0\\
	\end{pmatrix}
}

\begin{document}
	\maketitle
	
	\tableofcontents
	
	\newpage
	\section{Représentation par SO(3)}
	
	\subsection{Le groupe SO(3)}
	
	L'action de $R$ sur le vecteur $v$ s'obtient avec le produit matriciel:
	\[
		u=Rv
	\]
	Si $R$ est une rotation alors alors, elle préserve les longueurs des vecteurs.	
	\begin{eqnarray*}
		 |u|&=&|v|\\
		u^Tu&=&v^Tv\\
		(Rv)^TRv&=&v^Tv\\
		v^TR^TRv&=&v^Tv\\
	\end{eqnarray*}
	Cela implique que $R$ est une matrice orthogonale $R\in O(3)$:
	\[
		\boxed{R^TR=I}
	\]

	De plus on a :
	\[
	det(R^TR)=det(I) \Rightarrow det(R)^2= 1 \Rightarrow \boxed{det(R)=\pm 1}
	\]
	
	SO(3), le groupe des matrices orthogonales de déterminant 1 forme un sous-groupe de O(3). Le groupe des matrices orthogonales de déterminant -1 n'est pas un sous groupe puisque la composition de deux telles matrices est de déterminant 1.
	
	
	\newpage
	\subsection{Algèbre de Lie de SO(3)}
	
	Si on exprime les représentation matricielles des rotations dans les plans $yz,xz$ et $xy$ ou ce qui est équivalent pour l'espace de dimension 3, autour des axes $x,y$ et $z$, on a trivialement:
	
	\[	
	R(\vec{x},\phi)=
	\begin{pmatrix}
	1&0&0\\
	0&cos(\phi)&-sin(\phi)\\
	0&sin(\phi)&cos(\phi)\\	
	\end{pmatrix}\ 
	R(\vec{y},\phi)=
	\begin{pmatrix}
	cos(\phi)&0&sin(\phi)\\
	0&1&0\\
	-sin(\phi)&0&cos(\phi)\\
	\end{pmatrix}
	\  
	R(\vec{z},\phi)=
	\begin{pmatrix}
	cos(\phi)&-sin(\phi)&0\\
	sin(\phi)&cos(\phi)&0\\
	0&0&1\\
	\end{pmatrix}
	\]
	Les rotations infinitésimales d'ordre 1 s'exprime alors par:
	\[	
	R(\vec{x},\delta\phi)=
	\begin{pmatrix}
	1&0&0\\
	0&1&-\delta\phi\\
	0&\delta\phi&1\\	
	\end{pmatrix}= I + \delta\phi \Jx= I + \delta\phi J_x
	\]
	\[	
	R(\vec{y},\delta\phi)=
	\begin{pmatrix}
	1&0&\delta\phi\\
	0&1&0\\
	-\delta\phi&0&1\\
	\end{pmatrix}= I + \delta\phi \Jy= I + \delta\phi J_z
	\]
	\[	
	R(\vec{z},\delta\phi)=
	\begin{pmatrix}
	1&-\delta\phi&0\\
	\delta\phi&1&0\\
	0&0&1\\
	\end{pmatrix}= I + \delta\phi \Jz= I + \delta\phi J_z
	\]
	
	Maintenant, si on construit la rotation par une infinité d'action de la rotation infinitésimale on obtient une représentation par une exponentielle:
	
	\[	
		R(\vec{x},\phi) = \lim_{n \to \infty} R\left(\vec{x},\frac{\phi}{n}\right)^n= \lim_{n \to \infty} \left( I+\frac{\phi J_x}{n}\right)^n= e^{\phi J_x}
	\]	
	
	Cette formule se généralise pour un axe quelconque $\vec{n}$:
	
	\[
		\boxed{R(\vec{n},\phi) = e^{\vec{n}.\vec{J} \phi} \quad \quad avec\quad\quad \vec{J}=(J_x,J_y,J_z)^T }
	\]
	
	Maintenant on peut étudier les crochets de Lie :
	
	\[
		[J_x,J_y]=J_xJ_y-J_yJ_x=\Jx\Jy-\Jy\Jx=\Jz=J_z
	\]
	\[
		[J_y,J_z]=J_yJ_z-J_zJ_y=\Jy\Jz-\Jz\Jy=\Jx=J_x
	\]
	\[
		[J_z,J_x]=J_zJ_x-J_xJ_z=\Jz\Jx-\Jx\Jz=\Jy=J_y
	\]
	
	Finalement on a 
	
	\[
		\boxed{[J_i,J_j]=\sum_k \epsilon_{ijk}J_k}
	\]
	
	Avec $\epsilon_{ijk}$ le symbole de Levi-Civita:
	
	\[
		\epsilon_{ijk}=\left\{
		\begin{array}{l}
		+1\ Si\ (i,j,k)\ est \ (1,2,3)(2,3,1)(3,1,2)\\
		-1\ Si\ (i,j,k)\ est \ (3,2,1)(2,1,3)(1,3,2)\\
		0\ Sinon\\
		\end{array}
		\right.
	\]
	
	\subsection{Formule d'Euler-Rodrigues}
	Partons de :
	
	\[
		R(\vec{n},\phi) = e^{\vec{n}.\vec{J} \phi}
	\]
	
	Il s'agit de calculer l'exponentielle. Notons $M=\vec{n}.\vec{J}$
	
	\[
		\boxed{M=\floor{n}_{\times}=\begin{pmatrix}
			0&-n_z&n_y\\
			n_z&0&-n_x\\
			-n_y&n_x&0\\
			\end{pmatrix}}
		\quad\quad et \quad\quad \boxed{M\vec{v}=\vec{n} \times \vec{v}}
	\]
	
	On a 
	\[
	M^2=\begin{pmatrix}
	0&-n_z&n_y\\
	n_z&0&-n_x\\
	-n_y&n_x&0\\
	\end{pmatrix}\begin{pmatrix}
	0&-n_z&n_y\\
	n_z&0&-n_x\\
	-n_y&n_x&0\\
	\end{pmatrix}=\begin{pmatrix}
	-(n_z^2+n_y^2)&n_x n_y& n_x n_z\\
	n_x n_y&-(n_z^2+n_x^2)& n_y n_z\\
	n_x n_z&n_y n_z&-(n_y^2+n_x^2)\\
	\end{pmatrix}
	\]
	Et 
	\[
	M^3=\begin{pmatrix}
	-(n_z^2+n_y^2)&n_x n_y& n_x n_z\\
	n_x n_y&-(n_z^2+n_x^2)& n_y n_z\\
	n_x n_z&n_y n_z&-(n_y^2+n_x^2)\\
	\end{pmatrix}\begin{pmatrix}
	0&-n_z&n_y\\
	n_z&0&-n_x\\
	-n_y&n_x&0\\
	\end{pmatrix}
	\]
	\[
	M^3=(n_z^2+n_y^2+n_x^2)
	\begin{pmatrix}
	0&n_z&-n_y\\
	-n_z&0&n_x\\
	n_y&-n_x&0\\
	\end{pmatrix}
	\]
	\[
	\boxed{M^3=-M}
	\]
	
	\[
		\begin{array}{c|ccccccc}
		n=&0&1&2&3&4&5&6\\
		\hline
		M^n=&I&M&M^2&-M&-M^2&M&M^2\\
		\end{array}
	\]
	
	
	Donc $M^{2n+1}=(-1)^{n} M$ et $M^{2n}=(-1)^{n+1} M^2$ si $n>0$ sinon $M^0=I$ et donc on va pouvoir calculer explicitement l'exponentielle par sont développement de Taylor :
	
	\[
	e^{\vec{n}.\vec{J}\phi}=e^{M\phi} = \sum_{n=0}^{\infty} \frac{M^n\phi^n}{n!}=\sum_{n=0}^{\infty} \frac{M^{2n}\phi^{2n}}{(2n)!} + \sum_{n=0}^{\infty} \frac{M^{2n+1}\phi^{2n+1}}{(2n+1)!}
	\]
	
	\begin{eqnarray*}
	e^{\vec{n}.\vec{J}\phi}&=& I+M^2\sum_{n=1}^{\infty} \frac{(-1)^{n+1} \phi^{2n}}{(2n)!} + M\sum_{n=0}^{\infty} \frac{(-1)^{n}\phi^{2n+1}}{(2n+1)!}\\
	&=& I-M^2\sum_{n=1}^{\infty} \frac{(-1)^{n} \phi^{2n}}{(2n)!} + M\sum_{n=0}^{\infty} \frac{(-1)^{n}\phi^{2n+1}}{(2n+1)!}\\
	&=& I-M^2 (cos(\phi)-1) + M sin(\phi)\\
	\end{eqnarray*}
	
	Finalement 
	
	\[
	\boxed{R=e^{M\phi}=I+sin(\phi)M+(1-cos(\phi))M^2 }
	\]
	
	\newpage
	Si on applique à un vecteur:
	
	\begin{eqnarray*}
	\vec{v'}&=& R \vec{v} \\
	&=& (I+sin(\phi)M+(1-cos(\phi))M^2) \vec{v} \\
	&=& \vec{v}+sin(\phi)M\vec{v}+(1-cos(\phi))M M\vec{v}  \\
	&=& \vec{v}+sin(\phi)(\vec{n}\times \vec{v})+(1-cos(\phi))M (\vec{n}\times \vec{v})  \\
	&=& \vec{v}+sin(\phi)(\vec{n}\times \vec{v})+(1-cos(\phi)) (\vec{n}\times (\vec{n}\times \vec{v}))  \\
	\end{eqnarray*}
	
Mais $\vec{u}\times(\vec{v}\times \vec{w})=(\vec{u}.\vec{w})\vec{v}-(\vec{u}.\vec{v})\vec{w}$ donc :
    
    \begin{eqnarray*}
    \vec{n}\times(\vec{n}\times \vec{v})&=&(\vec{n}.\vec{v})\vec{n}-(\vec{n}.\vec{n})\vec{v}\\
	&=&(\vec{n}.\vec{v})\vec{n}-\vec{v}\\	
	\end{eqnarray*}

Donc 
	\begin{eqnarray*}
		\vec{v'}&=& \vec{v}+sin(\phi)(\vec{n}\times \vec{v})+(1-cos(\phi)) ((\vec{n}.\vec{v})\vec{n}-\vec{v})  \\
		&=& \vec{v}+sin(\phi)(\vec{n}\times \vec{v})+(1-cos(\phi)) ((\vec{n}.\vec{v})\vec{n})+cos(\phi)\vec{v}-\vec{v}  \\
	\end{eqnarray*}
Et finalement :
\[
\boxed{\vec{v'}= cos(\phi)\vec{v}+sin(\phi)(\vec{n}\times \vec{v})+(1-cos(\phi)) (\vec{n}.\vec{v})\vec{n} }
\]

	\newpage
	\subsection{Axe et angle de rotation}
	
Soit $R$ une matrice de rotation. Alors l'axe de rotation $n$ est un vecteur propre de cette matrice de valeur propre 1 et aussi de la matrice de rotation inverse. Donc :

\[
R\vec{n}=\vec{n} \qquad et\ aussi \qquad R^T\vec{n}=\vec{n}
\]  

Donc 

\[
\boxed{(R-R^T)\vec{n}=0}
\]

On peu alors remarque que $(R-R^T)$ est une matrice anti-symétrique de trace nulle. Et si on pose 
$(R-R^T)=k \floor{n}_{\times}$ alors l'équation précédente est bien vérifiée car $k \vec{n}\times \vec{n}=0$. Pour identifier $k$ on peut calculer explicitement $(R-R^T)$ avec la formule d'Euler-Rodrigues ce qui donne:

\begin{eqnarray*}
	(R-R^T)&=& (I +\sin(\phi) M+(1-\cos(\phi))M^2)-(I^T +\sin(\phi) M^T+(1-\cos(\phi))(M^2)^T)\\
	&\qquad Mais\qquad& M^T=-M\\
	&\qquad Et \qquad& M^2=(M^2)^T\\
	&=& 2 \sin(\phi) M\\
\end{eqnarray*}

Soit finalement:
\[
\boxed{(R-R^T)=2 \sin(\phi)\floor{n}_{\times}}
\]

Attention car il existe des matrices de rotation tel que $R=R^T$ qui correspondent au cas ou $sin(\phi)=0$, pour ces cas il est beaucoup plus stable numériquement de décomposer $R$ en valeur propre et de prendre le vecteur propre associée à la valeur propre de 1.\\

Maintenant nous avons aussi:

\[
	\boxed{tr(R)=1+2 \cos(\phi)}
\]

En effet :

\begin{eqnarray*}
tr(R)&=&tr(I)+(1-\cos(\phi)) tr(M^2)\\
&=&3-(1-\cos(\phi))(n_z^2+n_y^2+n_z^2+n_x^2+n_y^2+n_x^2)\\
&=&3-(1-\cos(\phi))(2)\\
&=&1+2 \cos(\phi)\\
\end{eqnarray*}
	
	\newpage
	\subsection{Composition des rotations (vecteurs colonnes)}
	
	Soit $A$ et $B$ deux repères orthonormés exprimés dans une base quelconque par des vecteurs colonnes:
	
	\[
		A=\left(\begin{array}{c|c|c}
		&&\\
		\vec{A_x}&\vec{A_y}&\vec{A_z}\\
		&&\\
		\end{array}
		\right)		
		\qquad 
		et
		\qquad
		B=\left(\begin{array}{c|c|c}
		&&\\
		\vec{B_x}&\vec{B_y}&\vec{B_z}\\
		&&\\
		\end{array}
		\right)
	\]
	
	
	Et soit $R_{AB}$ la matrice de rotation telle que:
	
	\[
	\boxed{B=A R_{AB}} \quad \quad donc \quad \quad \boxed{R_{AB}=A^TB}
	\]
	
	Les $R$ ainsi défini respectent la relation de Chasles. En effet si on introduit un troisième repère $C$ et $R_{AC}$,$R_{CB}$ les matrices de rotation telles que:
	
	\[
	C=A R_{AC}\quad et \quad B=C R_{CB}  \quad\quad Soit\quad \quad R_{AC}=A^TC \quad et\quad  R_{CB}=C^TB
	\]
	
	Alors 
	
	\[
	R_{AC}R_{CB}=A^TC C^TB = A^TB
	\]
	Et donc 
	\[
	\boxed{R_{AC}R_{CB}=R_{AB}}
	\]
	
	$R_{AB}$ sont les vecteurs de la base $B$ exprimés dans $A$ c'est à dire que:
	
	\[
		R_{AB}=A^TB=
		\left(\begin{array}{ccc}
		&\vec{A_x}&\\
		\hline
		&\vec{A_y}&\\
		\hline
		&\vec{A_z}&\\
		\end{array}
		\right)
		\left(\begin{array}{c|c|c}
		&&\\
		\vec{B_x}&\vec{B_y}&\vec{B_z}\\
		&&\\
		\end{array}
		\right)
		=
		\left(\begin{array}{ccc}
		\vec{A_x}.\vec{B_x}&\vec{A_x}.\vec{B_y}&\vec{A_x}.\vec{B_z}\\
		\vec{A_y}.\vec{B_x}&\vec{A_y}.\vec{B_y}&\vec{A_y}.\vec{B_z}\\
		\vec{A_z}.\vec{B_x}&\vec{A_z}.\vec{B_y}&\vec{A_z}.\vec{B_z}\\
		\end{array}
		\right)
	\]
	
	Maintenant, soit $V_{A}$ et $V_{B}$, des vecteurs exprimés respectivement dans les bases $A$ et $B$. Alors:
	
	\[
	A V_{A} =B V_{B} \quad\quad Soit\quad \quad V_{A} =A^T B V_{B}
	\]
	
	Par conséquent on a :
	
	\[
	\boxed{V_{A} = R_{AB} V_{B}} 
	\]
	
	\newpage
	\subsection{La composition des rotations (vecteurs lignes)}
	
	Soit $A$ et $B$ deux repères orthonormés exprimés dans une base quelconque par des vecteurs lignes. 
	
	\[
	A=\left(\begin{array}{ccc}
	&\vec{A_x}&\\
	\hline
	&\vec{A_y}&\\
	\hline
	&\vec{A_z}&\\
	\end{array}
	\right)		
	\qquad 
	et
	\qquad
	B=\left(\begin{array}{ccc}
	&\vec{B_x}&\\
	\hline
	&\vec{B_y}&\\
	\hline
	&\vec{B_z}&\\
	\end{array}
	\right)	
	\]
	
	Et soit $R_{AB}$ la matrice de rotation telle que:
	
	\[
	\boxed{A=R_{AB}B} \quad \quad donc \quad \quad \boxed{R_{AB}=AB^T}
	\]
	
	Les $R$ ainsi défini respectent la relation de Chasles, en effet si on introduit un troisième repère $C$ et $R_{AC}$,$R_{CB}$ les matrices de rotation telles que:
	
	\[
	A=R_{AC}C\quad et \quad C=R_{CB}B  \quad\quad Soit\quad \quad R_{AC}=AC^T \quad et\quad  R_{CB}=CB^T
	\]
	
	Alors 
	
	\[
	R_{AC}R_{CB}=AC^TCB^T = AB^T
	\]
	Et donc 
	\[
	\boxed{R_{AC}R_{CB}=R_{AB}}
	\]
	
	Maintenant, soit $V_{A}$ et $V_{B}$, des vecteurs exprimés respectivement dans les bases $A$ et $B$. Alors:
	
	\[
	V_{A} A  =V_{B} B  \quad\quad Soit\quad \quad V_{A} = V_{B} B A^T 
	\]
	
	Par conséquent on a :
	
	\[
	\boxed{V_{A} = V_{B} R_{BA} }
	\]
	
	
\newpage
\section{Représentation par SU(2)}
	\subsection{Les matrices de Pauli}
	 Introduisons les matrices $\sigma$ de Pauli:
	
	\[
	\sigma_x=\begin{pmatrix}
	0&1\\
	1&0\\
	\end{pmatrix}\quad
	\sigma_y=\begin{pmatrix}
	0&-i\\
	i&0\\
	\end{pmatrix}\quad
	\sigma_z=\begin{pmatrix}
	1&0\\
	0&-1\\
	\end{pmatrix}\quad et\quad \vec{\sigma}=(\sigma_x,\sigma_y,\sigma_z)^T
	\]
	
	On peut observer que le carré de ces matrices donne l'identité et que les matrices sont hermitiennes:  
	
	\[
		\boxed{\sigma_i^2=I}\quad\quad et \quad\quad \boxed{\sigma^{\dagger}=\sigma}
	\]
	
	En effet:
	\[
	\sigma_x^2=\begin{pmatrix}
	0&1\\
	1&0\\
	\end{pmatrix}
	\begin{pmatrix}
	0&1\\
	1&0\\
	\end{pmatrix}
	=
	\begin{pmatrix}
	1&0\\
	0&1\\
	\end{pmatrix}
	\]
	\[
	\sigma_y^2=\begin{pmatrix}
	0&-i\\
	i&0\\
	\end{pmatrix}
	\begin{pmatrix}
	0&-i\\
	i&0\\
	\end{pmatrix}
	=
	\begin{pmatrix}
	1&0\\
	0&1\\
	\end{pmatrix}
	\]
	\[
	\sigma_z^2=\begin{pmatrix}
	1&0\\
	0&-1\\
	\end{pmatrix}
	\begin{pmatrix}
	1&0\\
	0&-1\\
	\end{pmatrix}
	=
	\begin{pmatrix}
	1&0\\
	0&1\\
	\end{pmatrix}
	\]
	
	Et plus généralement l'anti-commutateur est nul si $i\ne j$:
	
	\[
		\boxed{ \{\sigma_i,\sigma_j\}=\sigma_i\sigma_j+\sigma_j\sigma_i=2 \delta_{ij} I}
	\]
	En effet:
	\[
	\sigma_x\sigma_y+\sigma_y\sigma_x=\begin{pmatrix}
	0&1\\
	1&0\\
	\end{pmatrix}
	\begin{pmatrix}
	0&-i\\
	i&0\\
	\end{pmatrix}
	+
	\begin{pmatrix}
	0&-i\\
	i&0\\
	\end{pmatrix}
	\begin{pmatrix}
	0&1\\
	1&0\\
	\end{pmatrix}	
	=
	\begin{pmatrix}
	i&0\\
	0&-i\\
	\end{pmatrix}
	+
	\begin{pmatrix}
	-i&0\\
	0&i\\
	\end{pmatrix}
	=0
	\]
	
	Mais pour les commutateurs on obtient:
	
	\[
	\boxed{[\sigma_i,\sigma_j]=2i \sum_k \epsilon_{ijk}\sigma_k}
	\]
	En effet:
	\[
	\sigma_x\sigma_y-\sigma_y\sigma_x=\begin{pmatrix}
	0&1\\
	1&0\\
	\end{pmatrix}
	\begin{pmatrix}
	0&-i\\
	i&0\\
	\end{pmatrix}
	-
	\begin{pmatrix}
	0&-i\\
	i&0\\
	\end{pmatrix}
	\begin{pmatrix}
	0&1\\
	1&0\\
	\end{pmatrix}	
	=
	\begin{pmatrix}
	i&0\\
	0&-i\\
	\end{pmatrix}
	-
	\begin{pmatrix}
	-i&0\\
	0&i\\
	\end{pmatrix}
	=2i \sigma_z
	\]
	
	Maintenant si on pose :
	
	\[
		\boxed{K_i=-\frac{i}{2}\sigma_i}
	\quad\quad alors \quad\quad
	[K_i,K_j]=-\frac{1}{4}[\sigma_i,\sigma_j]=-\frac{i}{2} \sum_k \epsilon_{ijk}\sigma_k = \sum_k \epsilon_{ijk}K_k
	\]
	
	Finalement les matrices $K_i$ satisfont le crochet de Lie de SO(3):
	\[
	\boxed{[K_i,K_j]= \sum_k \epsilon_{ijk}K_k}
	\]
	
\newpage
\subsection{Calcul de l'exponentielle pour SU(2)}

Tout groupe de 3 objets vérifiant l'algèbre de lie de SO(3) fournit une représentation du groupe de rotations. Donc les matrices $K_i$ fournissent une représentation du groupe de rotation par l'application exponentielle:

\[
\boxed{U(\vec{n},\phi) = e^{\vec{n}.\vec{K} \phi}= e^{-\frac{i}{2}\vec{n}.\vec{\sigma} \phi}}
\]

On remarque alors que les rotations sont représentées p par une matrice unitaire $U$:

\[
\boxed{U^{\dagger}U=I}
\] 

En effet:

\begin{eqnarray*}
U^{\dagger}U&=&\left(e^{-\frac{i}{2}\vec{n}.\vec{\sigma} \phi}\right)^\dagger e^{-\frac{i}{2}\vec{n}.\vec{\sigma} \phi}\\
&=&e^{\frac{i}{2}\vec{n}.\vec{\sigma^\dagger} \phi} e^{-\frac{i}{2}\vec{n}.\vec{\sigma} \phi}\\
&=&e^{\frac{i}{2}\vec{n}.\vec{\sigma} \phi} e^{-\frac{i}{2}\vec{n}.\vec{\sigma} \phi}\\
&=&I\\
\end{eqnarray*}

De manière analogue à ce que nous avions fait pour établir la formule de Rodrigue. posons :

\[
	M=\vec{n}.\vec{\sigma}=(n_x \sigma_x+n_y \sigma_y+n_z \sigma_z)= \begin{pmatrix}
	n_z&n_x-i n_y\\
	n_x+i n_y&-n_z\\
	\end{pmatrix}
\]

Calculons $M^2$:

\[
	M^2=\begin{pmatrix}
	n_z&n_x-i n_y\\
	n_x+i n_y&-n_z\\
	\end{pmatrix}
	\begin{pmatrix}
	n_z&n_x-i n_y\\
	n_x+i n_y&-n_z\\
	\end{pmatrix}
	=
	\begin{pmatrix}
	n_x^2+ n_y^2+n_z^2&0\\
	0&n_x^2+ n_y^2+n_z^2\\
	\end{pmatrix}
\]

Donc comme $|\vec{n}|=1$ on obtient que $\boxed{M^2=I}$. Calculons maintenant l'exponentielle:

\begin{eqnarray*}
U(\vec{n},\phi) &=& e^{-\frac{i}{2} M \phi}\\
&=& \sum_{n=0}^{\infty} \frac{(-i M \frac{\phi}{2})^n}{n!} \\
&=& \sum_{n=0}^{\infty} \frac{(-i M \frac{\phi}{2})^{2n}}{(2n)!} +\sum_{n=0}^{\infty} \frac{(-i M \frac{\phi}{2})^{2n+1}}{(2n+1)!} \\
&=& I \sum_{n=0}^{\infty} \frac{(-1)^n (\frac{\phi}{2})^{2n}}{(2n)!} +M\sum_{n=0}^{\infty} \frac{(-i)^{2n+1} (\frac{\phi}{2})^{2n+1}}{(2n+1)!} \\
&=& I \sum_{n=0}^{\infty} \frac{(-1)^n (\frac{\phi}{2})^{2n}}{(2n)!}-i M\sum_{n=0}^{\infty} \frac{(-1)^{n} (\frac{\phi}{2})^{2n+1}}{(2n+1)!} \\
&=& I \cos\left(\frac{\phi}{2}\right)-i M \sin\left(\frac{\phi}{2}\right)\\
\end{eqnarray*}

Finalement :

\[
\boxed{U(\vec{n},\phi)= I \cos\left(\frac{\phi}{2}\right)-i \vec{n}.\vec{\sigma} \sin\left(\frac{\phi}{2}\right)}
\]

\newpage
\subsection{Action de SU(2)}

A chaque vecteur $\vec{r}$ de $\mathbb{R}^3$, on peut faire correspondre la matrice $X$ tel que:

\[
	\boxed{X=\vec{r}.\vec{\sigma}} \quad soit \quad X=\begin{pmatrix}
	r_z&r_x-i r_y\\
	r_x+i r_y&-r_z\\
	\end{pmatrix}
\]

Calculons le produit de deux tel matrices:
\begin{eqnarray*}
(\vec{a}.\vec{\sigma}) (\vec{b}.\vec{\sigma})&=&\begin{pmatrix}
a_z&a_x-i a_y\\
a_x+i a_y&-a_z\\
\end{pmatrix}
\begin{pmatrix}
b_z&b_x-i b_y\\
b_x+i b_y&-b_z\\
\end{pmatrix}\\
&=&
\begin{pmatrix}
(b_x a_x+ b_y a_y+a_z b_z)  +i (b_y a_x- b_x a_y)& (b_x a_z- b_z a_x) + i (b_z a_y- b_y a_z)  \\
(b_z a_x  - b_x a_z) + i(b_z a_y-b_y a_z)& (b_x a_x +b_y a_y+ b_z a_z)+ i (b_x a_y - b_y a_x )\\
\end{pmatrix}\\
&=&
I \vec{a}.\vec{b}+ i\begin{pmatrix}
(b_y a_x- b_x a_y)&   (b_z a_y- b_y a_z)-i(b_x a_z- b_z a_x)  \\
 (b_z a_y-b_y a_z)+i(b_x a_z - b_z a_x)& (b_x a_y - b_y a_x )\\
\end{pmatrix}
\end{eqnarray*}

Et si on pose :

\[
\vec{c}=\begin{pmatrix}
b_z a_y- b_y a_z\\
b_x a_z- b_z a_x\\
b_y a_x- b_x a_y\\
\end{pmatrix}= \vec{a}\times\vec{b}
\]

Donc finalement on a la relation :

\[
 \boxed{ (\vec{a}.\vec{\sigma}) (\vec{b}.\vec{\sigma}) = \vec{a}.\vec{b} I + i(\vec{a}\times\vec{b}).\vec{\sigma}}
\]

On peut inverser la carte par la relation:

\[
\boxed{r_i= \frac{1}{2} Tr(X.\sigma_i)}
\]

On peut alors agir sur $X$ par :

\[
	\boxed{ X'=U X U^{\dagger} }
\]

Ce qui donne en s'armant de courage:
\begin{eqnarray*}
	\vec{r'}.\vec{\sigma}&=&U (\vec{r}.\vec{\sigma}) U^{\dagger}\\
	&=&\left(I \cos\left(\frac{\phi}{2}\right)-i \vec{n}.\vec{\sigma} \sin\left(\frac{\phi}{2}\right)\right) (\vec{r}.\vec{\sigma}) \left(I \cos\left(\frac{\phi}{2}\right)-i \vec{n}.\vec{\sigma} \sin\left(\frac{\phi}{2}\right)\right)^{\dagger}\\
	&=&\left(I \cos\left(\frac{\phi}{2}\right)-i \vec{n}.\vec{\sigma} \sin\left(\frac{\phi}{2}\right)\right) (\vec{r}.\vec{\sigma}) \left(I \cos\left(\frac{\phi}{2}\right)+i \vec{n}.\vec{\sigma} \sin\left(\frac{\phi}{2}\right)\right)\\
	&=&\left((\vec{r}.\vec{\sigma}) \cos\left(\frac{\phi}{2}\right)-i (\vec{n}.\vec{\sigma})(\vec{r}.\vec{\sigma}) \sin\left(\frac{\phi}{2}\right)\right)  \left(I \cos\left(\frac{\phi}{2}\right)+i \vec{n}.\vec{\sigma} \sin\left(\frac{\phi}{2}\right)\right)\\
	&=&\left((\vec{r}.\vec{\sigma}) \cos^2\left(\frac{\phi}{2}\right)-i (\vec{n}.\vec{\sigma})(\vec{r}.\vec{\sigma}) \sin\left(\frac{\phi}{2}\right)\cos\left(\frac{\phi}{2}\right)\right) + \\& & \left(i(\vec{r}.\vec{\sigma})(\vec{n}.\vec{\sigma}) \cos\left(\frac{\phi}{2}\right)\sin\left(\frac{\phi}{2}\right)+ (\vec{n}.\vec{\sigma})(\vec{r}.\vec{\sigma})(\vec{n}.\vec{\sigma}) \sin^2\left(\frac{\phi}{2}\right)\right)\\
	&=&\left((\vec{r}.\vec{\sigma}) \cos^2\left(\frac{\phi}{2}\right)-i (\vec{n}.\vec{r} I + i(\vec{n}\times\vec{r}).\vec{\sigma}) \sin\left(\frac{\phi}{2}\right)\cos\left(\frac{\phi}{2}\right)\right) + \\& & \left(i(\vec{r}.\vec{n} I + i(\vec{r}\times\vec{n}).\vec{\sigma}) \cos\left(\frac{\phi}{2}\right)\sin\left(\frac{\phi}{2}\right)+ (\vec{n}.\vec{\sigma})(\vec{r}.\vec{\sigma})(\vec{n}.\vec{\sigma}) \sin^2\left(\frac{\phi}{2}\right)\right)\\
	&=&(\vec{r}.\vec{\sigma}) \cos^2\left(\frac{\phi}{2}\right)-i (\vec{n}.\vec{r} I + i(\vec{n}\times\vec{r}).\vec{\sigma}-\vec{r}.\vec{n} I - i(\vec{r}\times\vec{n}).\vec{\sigma}) \sin\left(\frac{\phi}{2}\right)\cos\left(\frac{\phi}{2}\right) + \\& & (\vec{n}.\vec{\sigma})(\vec{r}.\vec{\sigma})(\vec{n}.\vec{\sigma}) \sin^2\left(\frac{\phi}{2}\right)\\
	&=&\frac{1}{2}(\vec{r}.\vec{\sigma}) (\cos(\phi)+1)+ 2((\vec{n}\times\vec{r}).\vec{\sigma}) \sin\left(\frac{\phi}{2}\right)\cos\left(\frac{\phi}{2}\right) + (\vec{n}.\vec{\sigma})(\vec{r}.\vec{n} I + i(\vec{r}\times\vec{n}).\vec{\sigma}) \sin^2\left(\frac{\phi}{2}\right)\\
	&=&\frac{1}{2}(\vec{r}.\vec{\sigma}) (\cos(\phi)+1)+ 2((\vec{n}\times\vec{r}).\vec{\sigma}) \sin\left(\frac{\phi}{2}\right)\cos\left(\frac{\phi}{2}\right) + (\vec{r}.\vec{n})(\vec{n}.\vec{\sigma})\sin^2\left(\frac{\phi}{2}\right)+ i(\vec{n}.\vec{\sigma})((\vec{r}\times\vec{n}).\vec{\sigma}) \sin^2\left(\frac{\phi}{2}\right)\\
	&=&\left(\frac{1}{2} (\cos(\phi)+1) \vec{r} + \sin(\phi)(\vec{n}\times\vec{r}) + \sin^2\left(\frac{\phi}{2}\right)(\vec{r}.\vec{n})\vec{n}\right).\vec{\sigma}+ i(\vec{n}.\vec{\sigma})((\vec{r}\times\vec{n}).\vec{\sigma}) \sin^2\left(\frac{\phi}{2}\right)\\
	&=&\left(\frac{1}{2} (\cos(\phi)+1) \vec{r} + \sin(\phi)(\vec{n}\times\vec{r}) + \sin^2\left(\frac{\phi}{2}\right)(\vec{r}.\vec{n})\vec{n}\right).\vec{\sigma}+ i( \vec{n}.(\vec{r}\times \vec{n}) I + i(\vec{n}\times (\vec{r}\times \vec{n})).\vec{\sigma} ) \sin^2\left(\frac{\phi}{2}\right)\\
	&=&\left(\frac{1}{2} (\cos(\phi)+1) \vec{r} + \sin(\phi)(\vec{n}\times\vec{r}) + \sin^2\left(\frac{\phi}{2}\right)(\vec{r}.\vec{n})\vec{n}\right).\vec{\sigma}-( ((\vec{n}.\vec{n})\vec{r} - (\vec{n}.\vec{r})\vec{n}).\vec{\sigma} )	
	\sin^2\left(\frac{\phi}{2}\right)\\	
	&=&\left(\frac{1}{2} (\cos(\phi)+1) \vec{r} + \sin(\phi)(\vec{n}\times\vec{r}) + \frac{1}{2}(1-\cos(\phi)) (-\vec{r} + 2 (\vec{n}.\vec{r})\vec{n} ) \right).\vec{\sigma}\\	
	&=&\left(\cos(\phi) \vec{r}+\frac{1}{2}(1-\cos(\phi))\vec{r} + \sin(\phi)(\vec{n}\times\vec{r}) + \frac{1}{2}(1-\cos(\phi)) (-\vec{r} + 2 (\vec{n}.\vec{r})\vec{n}  ) \right).\vec{\sigma}\\
	&=&\left(\cos(\phi) \vec{r} + \sin(\phi)(\vec{n}\times\vec{r}) + (1-\cos(\phi)) ( (\vec{n}.\vec{r})\vec{n}  ) \right).\vec{\sigma}\\
\end{eqnarray*}

Ce qui donne bien :

\[
\boxed{r'=\cos(\phi) \vec{r} + \sin(\phi)(\vec{n}\times\vec{r}) + (1-\cos(\phi)) ( (\vec{n}.\vec{r})\vec{n}  )}
\]

\newpage
\section{Annexes}
\subsection{Retrouver l'identité à partir de la formule d'Euler-Rodrigue}
Avec $M^T=-M$ et $(M^2)^T=M^2$ et $M^3=-M$ on verifie que:

\begin{eqnarray*}
	R^T R&=&(I+sin(\phi)M+(1-cos(\phi))M^2)^T (I+sin(\phi)M+(1-cos(\phi))M^2)\\
	&=&I+sin(\phi)M^T+(1-cos(\phi))(M^2)^T+\\
	&&(I+sin(\phi)M^T+(1-cos(\phi))(M^2)^T)sin(\phi)M+\\
	&&(I+sin(\phi)M^T+(1-cos(\phi))(M^2)^T)(1-cos(\phi))M^2\\
	&=&I+sin(\phi)M^T+(1-cos(\phi))(M^2)^T+\\
	&&sin(\phi)M+sin(\phi)sin(\phi)M^TM+(1-cos(\phi))sin(\phi)(M^2)^TM+\\
	&&(1-cos(\phi))M^2+sin(\phi)(1-cos(\phi))M^TM^2+(1-cos(\phi))^2(M^2)^TM^2\\
	&=&I-sin(\phi)M+(1-cos(\phi))M^2+\\
	&&sin(\phi)M-sin(\phi)sin(\phi)M^2-(1-cos(\phi))sin(\phi)M+\\
	&&(1-cos(\phi))M^2-sin(\phi)(1-cos(\phi))M^3-(1-cos(\phi))^2M^2\\
	&=&I+(2(1-cos(\phi))-sin(\phi)^2-(1-cos(\phi))^2)M^2\\
	&=&I+(2-2cos(\phi)-sin(\phi)^2-1+2cos(\phi)-cos(\phi)^2)M^2\\
	&=&I+(1-sin(\phi)^2-cos(\phi)^2)M^2\\
	&=&I\\	
\end{eqnarray*}

\subsection{Erreur entre deux rotations (en cours...)}

Soit $R_a$ et $R_b$ deux rotations d'angles $a$ et $b$ et de vecteur $A$ et $B$. On a alors 4 possibilités de calculer une rotation d'erreur entre $R_a$ et $R_b$.

\[
\boxed{E_1=R_a^T R_b \qquad
E_2=R_b^T R_a \qquad
E_3=R_a R_b^T \qquad
E_4=R_b R_a^T \qquad}
\]

On peut observer que $E_1=E_2^T$ et $E_3=E_4^T$ donc ces rotations on le même angle et des axes opposés:

\[
\boxed{\begin{array}{c}
angle(E_1)=angle(E_2)\qquad  et\qquad angle(E_3)=angle(E_4) \\
axe(E_1)=-axe(E_2)\qquad  et\qquad axe(E_3)=-axe(E_4)\\
\end{array}}
\]

De plus, si les erreurs sont petites, alors la composition des rotations est commutative et dans ce cas $E_1=E_4$ et $E_2=E_3$. 

\[
\boxed{\begin{array}{c}
angle(E_1)=angle(E_2)=angle(E_3)=angle(E_4)\\
axe(E_1)=-axe(E_2)=-axe(E_3)=axe(E_4)\\
\end{array}
\qquad pour\ de\ petites\ rotations}
\]

Cependant lorsque les erreurs sont importantes $E_1\ne E_4$, si on représente $R_a$ et $R_b$ par $U_a$ et $U_b$ , deux éléments de $SU(2)$, calculons:

\[
E_1=U_a^\dagger U_b \qquad E_3= U_b U_a^\dagger
\]
Avec 
\[
U_a = I \cos\left(\frac{a}{2}\right)-i \vec{A}.\vec{\sigma} \sin\left(\frac{a}{2}\right)\qquad\qquad
U_b = I \cos\left(\frac{b}{2}\right)-i \vec{B}.\vec{\sigma} \sin\left(\frac{b}{2}\right)
\]

Commençons par $E_1$ :

\begin{eqnarray*}
E_1&=&\left( I \cos\left(\frac{a}{2}\right)+i \vec{A}.\vec{\sigma} \sin\left(\frac{a}{2}\right) \right)\left( I \cos\left(\frac{b}{2}\right)-i \vec{B}.\vec{\sigma} \sin\left(\frac{b}{2}\right) \right)\\
&=& I \cos\left(\frac{a}{2}\right)\cos\left(\frac{b}{2}\right)+i \vec{A}.\vec{\sigma} \sin\left(\frac{a}{2}\right)\cos\left(\frac{b}{2}\right)  -i  \vec{B}.\vec{\sigma} \cos\left(\frac{a}{2}\right)\sin\left(\frac{b}{2}\right)+ \vec{A}.\vec{\sigma}\vec{B}.\vec{\sigma} \sin\left(\frac{a}{2}\right)\sin\left(\frac{b}{2}\right) \\
\end{eqnarray*}

Mais 
\[
\begin{array}{l}
\sin(\frac{a}{2})cos(\frac{b}{2})=\frac{1}{2}\left(\sin\left(\frac{a-b}{2}\right)+\sin\left(\frac{a+b}{2}\right)\right)\\
\cos(\frac{a}{2})sin(\frac{b}{2})=\frac{1}{2}\left(\sin\left(\frac{a+b}{2}\right)-\sin\left(\frac{a-b}{2}\right)\right)\\
\cos(\frac{a}{2})cos(\frac{b}{2})=\frac{1}{2}\left(\cos\left(\frac{a-b}{2}\right)+\cos\left(\frac{a+b}{2}\right)\right)\\
\sin(\frac{a}{2})sin(\frac{b}{2})=\frac{1}{2}\left(\cos\left(\frac{a-b}{2}\right)-\cos\left(\frac{a+b}{2}\right)\right)\\
(\vec{A}.\vec{\sigma}) (\vec{B}.\vec{\sigma}) = \left(\vec{A}.\vec{B} I + i(\vec{A}\times\vec{B}).\vec{\sigma}\right)\\
\end{array}
\]

\begin{eqnarray*}
	E_1&=& I \frac{1}{2}\left(\cos\left(\frac{a-b}{2}\right)+\cos\left(\frac{a+b}{2}\right)\right)\\
	&&+i \vec{A}.\vec{\sigma} \frac{1}{2}\left(\sin\left(\frac{a-b}{2}\right)+\sin\left(\frac{a+b}{2}\right)\right)\\
	&&-i  \vec{B}.\vec{\sigma} \frac{1}{2}\left(\sin\left(\frac{a+b}{2}\right)-\sin\left(\frac{a-b}{2}\right)\right)\\
	&&+ \left(\vec{A}.\vec{B} I + i(\vec{A}\times\vec{B}).\vec{\sigma}\right) \frac{1}{2}\left(\cos\left(\frac{a-b}{2}\right)-\cos\left(\frac{a+b}{2}\right)\right) \\
	2 E_1&=& I \left((1+\vec{A}.\vec{B})\cos\left(\frac{a-b}{2}\right)+(1-\vec{A}.\vec{B})\cos\left(\frac{a+b}{2}\right)\right)\\
	&&+i \left( \vec{A}.\vec{\sigma}\sin\left(\frac{a-b}{2}\right)+ \vec{A}.\vec{\sigma}\sin\left(\frac{a+b}{2}\right)\right)\\
	&&+i \left(  -\vec{B}.\vec{\sigma}\sin\left(\frac{a+b}{2}\right)+  \vec{B}.\vec{\sigma}\sin\left(\frac{a-b}{2}\right)\right)\\
	&&+i (\vec{A}\times\vec{B}).\vec{\sigma} \left(\cos\left(\frac{a-b}{2}\right)-\cos\left(\frac{a+b}{2}\right)\right) \\
	2 E_1&=& I \left((1+\vec{A}.\vec{B})\cos\left(\frac{a-b}{2}\right)+(1-\vec{A}.\vec{B})\cos\left(\frac{a+b}{2}\right)\right)\\
	&&+i \vec{\sigma}. \left( (\vec{A}+\vec{B})\sin\left(\frac{a-b}{2}\right)+ (\vec{A}-\vec{B})\sin\left(\frac{a+b}{2}\right)+(\vec{A}\times\vec{B}) \left(\cos\left(\frac{a-b}{2}\right)-\cos\left(\frac{a+b}{2}\right)\right)\right)\\
\end{eqnarray*}

\subsection{Logarithme of a rotation }

Prouvons que :
\[
log(R)=\frac{\phi}{2 sin(\phi)}(R-R^T)
\]

\[
||log(R)||_F=\sqrt(2)|\phi|
\]

\[
||log(A^TB)||_F=d_g(A,B)
\]
is the geodesic distance on the 3D manifold of rotation matrices

\end{document}