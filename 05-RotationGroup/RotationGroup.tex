

\documentclass[12pt,a4paper]{article}
\usepackage{geometry}
\usepackage{amsmath}
\usepackage{amsfonts}
\usepackage{graphicx}
\geometry{hmargin=1.0cm,vmargin=2.0cm}
\setlength{\parindent}{0cm}
\title{Le Groupe des Rotations}
\author{Loïc Huguel}

\usepackage{mathtools}
\usepackage{dsfont}
\usepackage{bm}
\DeclarePairedDelimiter\ceil{\lceil}{\rceil}
\DeclarePairedDelimiter\floor{\lfloor}{\rfloor}
\DeclareMathOperator*{\argmax}{arg\,max}
\DeclareMathOperator*{\argmin}{arg\,min}

\newcommand{\unit}
{
\bm{\mathds{1}}
}

\newcommand{\Jx}
{
\begin{pmatrix}
	0&0&0\\
	0&0&-1\\
	0&1&0\\
\end{pmatrix}
}

\newcommand{\Jy}
{
	\begin{pmatrix}
		0&0&1\\
		0&0&0\\
		-1&0&0\\
	\end{pmatrix}
}

\newcommand{\Jz}
{
	\begin{pmatrix}
		0&-1&0\\
		1&0&0\\
		0&0&0\\
	\end{pmatrix}
}

\begin{document}
	\maketitle
	
	\tableofcontents
	
	\newpage
	\section{Représentation par $SO(3)$}
	
	\subsection{Le groupe $SO(3)$}
	
	Si on souhaite définir une rotation par une matrice et que on définit l'action de $\bm{R}$ sur le vecteur $\bm{v}$ par le produit matriciel:
	\[
		\bm{u}=\bm{R}\bm{v}
	\]
	Si $\bm{R}$ est une rotation alors par définition, elle préserve les longueurs des vecteurs. C'est à dire:
	\begin{eqnarray*}
		 |\bm{u}|&=&|\bm{v}|\\
		\bm{u}^T\bm{u}&=&\bm{v}^T\bm{v}\\
		(\bm{R}\bm{v})^T\bm{R}\bm{v}&=&\bm{v}^T\bm{v}\\
		\bm{v}^T\bm{R}^T\bm{R}\bm{v}&=&\bm{v}^T\bm{v}\\
	\end{eqnarray*}
	Cela implique que $\bm{R}$ est une matrice orthogonale. Ces matrices formes un groupe noté  $O(n)$. On a donc $\bm{R}\in O(n)$:
	\[
		\boxed{\bm{R}^T\bm{R}=\mathds{\bm{1}} \quad \Rightarrow \quad \bm{R}^T=\bm{R}^{-1}}
	\]

	Sur le déterminant on a :
	\[
	det(\bm{R}^T\bm{R})=det(\mathds{\bm{1}}) \Rightarrow det(\bm{R})^2= 1 \Rightarrow \boxed{det(\bm{R})=\pm 1}
	\]
	
	Une rotation conserve également les orientations. Au sens ou si $\bm{E}=(\bm{e_1}|\bm{e_2}|...|\bm{e_n})$ forme un volume orienté de dans un sens direct/indirect alors les vecteurs $\bm{R} \bm{E}=(\bm{R}\bm{e_1}|\bm{R}\bm{e_2}|...|\bm{R}\bm{e_n})$ forme également un volume orienté dans un sens direct/indirect, c'est à dire que $det(\bm{E})=det(\bm{R}\bm{E})$. Il en découle que $det(\bm{R})=1$.\\
		
	Le groupe des matrices orthogonales de déterminant 1, noté $SO(n)$, forme un sous-groupe de $O(n)$. Le groupe des matrices orthogonales de déterminant -1 n'est pas un sous groupe puisque la composition de deux telles matrices est de déterminant 1.\\
	
	
	Sur $\mathbb{R}^3$ ou la notion de produit vectoriel a un sens, on a pour tout couple de vecteurs $(u,v)$:
	
	\[
		\boxed{\bm{R}(\bm{u} \times \bm{v})=\bm{R}\bm{u} \times \bm{R}\bm{v}}
	\]
	
	Ce qui traduit l'impossibilité de passer d'un repère direct à un repère indirect par une rotation.\\
	
	Comme nous n'avons pas fait de supposition sur la dimension de l'espace, tout ce que on vient de dire est valable sur $\mathbb{R}^n$.
	
	\newpage
	\subsection{Algèbre de Lie de $SO(3)$}
	
	Si on exprime les représentation matricielles des rotations dans les plans $\bm{yz},\bm{xz}$ et $\bm{xy}$ ou ce qui est équivalent pour l'espace de dimension 3, autour des axes $\bm{x},\bm{y}$ et $\bm{z}$, on a trivialement:
	
	\[	
	R(\bm{x},\phi)=
	\begin{pmatrix}
	1&0&0\\
	0&cos(\phi)&-sin(\phi)\\
	0&sin(\phi)&cos(\phi)\\	
	\end{pmatrix}\ 
	R(\bm{y},\phi)=
	\begin{pmatrix}
	cos(\phi)&0&sin(\phi)\\
	0&1&0\\
	-sin(\phi)&0&cos(\phi)\\
	\end{pmatrix}
	\  
	R(\bm{z},\phi)=
	\begin{pmatrix}
	cos(\phi)&-sin(\phi)&0\\
	sin(\phi)&cos(\phi)&0\\
	0&0&1\\
	\end{pmatrix}
	\]
	Les rotations infinitésimales d'ordre 1 s'exprime alors par:
	\[	
	R(\bm{x},\delta)=
	\begin{pmatrix}
	1&0&0\\
	0&1&-\delta\\
	0&\delta&1\\	
	\end{pmatrix}= \unit + \delta \Jx= \unit + \delta \bm{J_x}
	\]
	\[	
	R(\bm{y},\delta)=
	\begin{pmatrix}
	1&0&\delta\\
	0&1&0\\
	-\delta&0&1\\
	\end{pmatrix}= \unit + \delta \Jy= \unit + \delta \bm{J_y}
	\]
	\[	
	R(\bm{z},\delta)=
	\begin{pmatrix}
	1&-\delta&0\\
	\delta&1&0\\
	0&0&1\\
	\end{pmatrix}= \unit + \delta \Jz= \unit + \delta \bm{J_z}
	\]
	
	Maintenant, si on construit la rotation par une infinité d'action de la rotation infinitésimale on obtient une représentation par une exponentielle:
	
	\[	
		R(\bm{x},\phi) = \lim_{n \to \infty} R\left(\bm{x},\frac{\phi}{n}\right)^n= \lim_{n \to \infty} \left( \unit+\frac{\phi \bm{J_x}}{n}\right)^n= e^{\phi \bm{J_x}}
	\]	
	
	Cette formule se généralise pour un axe quelconque $\bm{n}$:
	
	\[
		\boxed{R(\bm{n},\phi) = e^{\bm{n}.\bm{J} \phi} \quad \quad avec\quad\quad \bm{J}=(\bm{J_x},\bm{J_y},\bm{J_z})^T }
	\]
	
	Les $\bm{J}$ sont appelés les générateurs infinitésimaux des rotations. Maintenant on peut calculer les crochets de Lie c'est à dire les commutateurs, on trouve:
	
	\[
		[\bm{J_x},\bm{J_y}]=\bm{J_x}\bm{J_y}-\bm{J_y}\bm{J_x}=\Jx\Jy-\Jy\Jx=\Jz=\bm{J_z}
	\]
	\[
		[\bm{J_y},\bm{J_z}]=\bm{J_y}\bm{J_z}-\bm{J_z}\bm{J_y}=\Jy\Jz-\Jz\Jy=\Jx=\bm{J_x}
	\]
	\[
		[\bm{J_z},\bm{J_x}]=\bm{J_z}\bm{J_x}-\bm{J_x}\bm{J_z}=\Jz\Jx-\Jx\Jz=\Jy=\bm{J_y}
	\]
	
	Finalement on a 
	
	\[
		\boxed{[\bm{J_i},\bm{J_j}]=\sum_k \epsilon_{ijk}\bm{J_k}}
	\]
	
	Avec $\epsilon_{ijk}$ le symbole de Levi-Civita:
	
	\[
		\epsilon_{ijk}=\left\{
		\begin{array}{l}
		+1\ Si\ (i,j,k)\ est \ (1,2,3)(2,3,1)(3,1,2)\\
		-1\ Si\ (i,j,k)\ est \ (3,2,1)(2,1,3)(1,3,2)\\
		0\ Sinon\\
		\end{array}
		\right.
	\]
	
	\subsection{Formule d'Euler-Rodrigues}
	Partons de :
	
	\[
		R(\bm{n},\phi) = e^{\bm{n}.\bm{J} \phi}
	\]
	
	Il s'agit de calculer l'exponentielle. Notons $\bm{M}=\bm{n}.\bm{J}$
	
	\[
		\boxed{\bm{M}=\floor{\bm{n}}_{\times}=\begin{pmatrix}
			0&-n_z&n_y\\
			n_z&0&-n_x\\
			-n_y&n_x&0\\
			\end{pmatrix}}
		\quad\quad et \quad\quad \boxed{\bm{M}\bm{v}=\bm{n} \times \bm{v}}
	\]
	
	On a 
	\[
	\bm{M}^2=\begin{pmatrix}
	0&-n_z&n_y\\
	n_z&0&-n_x\\
	-n_y&n_x&0\\
	\end{pmatrix}\begin{pmatrix}
	0&-n_z&n_y\\
	n_z&0&-n_x\\
	-n_y&n_x&0\\
	\end{pmatrix}=\begin{pmatrix}
	-(n_z^2+n_y^2)&n_x n_y& n_x n_z\\
	n_x n_y&-(n_z^2+n_x^2)& n_y n_z\\
	n_x n_z&n_y n_z&-(n_y^2+n_x^2)\\
	\end{pmatrix}
	\]
	Et 
	\[
	\bm{M}^3=\begin{pmatrix}
	-(n_z^2+n_y^2)&n_x n_y& n_x n_z\\
	n_x n_y&-(n_z^2+n_x^2)& n_y n_z\\
	n_x n_z&n_y n_z&-(n_y^2+n_x^2)\\
	\end{pmatrix}\begin{pmatrix}
	0&-n_z&n_y\\
	n_z&0&-n_x\\
	-n_y&n_x&0\\
	\end{pmatrix}
	\]
	\[
	\bm{M}^3=(n_z^2+n_y^2+n_x^2)
	\begin{pmatrix}
	0&n_z&-n_y\\
	-n_z&0&n_x\\
	n_y&-n_x&0\\
	\end{pmatrix}
	\]
	\[
	\boxed{\bm{M}^3=-\bm{M}}
	\]
	
	\[
		\begin{array}{|c|ccccccc|}
		\hline
		n=&0&1&2&3&4&5&6\\
		\hline
		\bm{M}^n=&\unit&\bm{M}&\bm{M}^2&-\bm{M}&-\bm{M}^2&\bm{M}&\bm{M}^2\\
		\hline
		\end{array}
	\]
	
	
	Donc $\bm{M}^{2n+1}=(-1)^{n} \bm{M}$ et $\bm{M}^{2n}=(-1)^{n+1} \bm{M}^2$ si $n>0$ sinon $\bm{M}^0=\unit$ et donc on va pouvoir calculer explicitement l'exponentielle par sont développement de Taylor :
	
	\[
	e^{\bm{n}.\bm{J}\phi}=e^{\bm{M}\phi} = \sum_{n=0}^{\infty} \frac{\bm{M}^n\phi^n}{n!}=\sum_{n=0}^{\infty} \frac{\bm{M}^{2n}\phi^{2n}}{(2n)!} + \sum_{n=0}^{\infty} \frac{\bm{M}^{2n+1}\phi^{2n+1}}{(2n+1)!}
	\]
	
	\begin{eqnarray*}
	e^{\bm{n}.\bm{J}\phi}&=& \unit+\bm{M}^2\sum_{n=1}^{\infty} \frac{(-1)^{n+1} \phi^{2n}}{(2n)!} + \bm{M}\sum_{n=0}^{\infty} \frac{(-1)^{n}\phi^{2n+1}}{(2n+1)!}\\
	&=& \unit-\bm{M}^2\sum_{n=1}^{\infty} \frac{(-1)^{n} \phi^{2n}}{(2n)!} + \bm{M}\sum_{n=0}^{\infty} \frac{(-1)^{n}\phi^{2n+1}}{(2n+1)!}\\
	&=& \unit-\bm{M}^2 (cos(\phi)-1) + \bm{M} sin(\phi)\\
	\end{eqnarray*}
	
	Finalement on obtient la formule d'Euler-Rodrigues:
	
	\[
	\boxed{\bm{R}=e^{\bm{M}\phi}=\unit+sin(\phi)\bm{M}+(1-cos(\phi))\bm{M}^2 }
	\]
	
	\newpage
	\subsection{Formule d'Euler-Rodrigues vectorielle}
	
	Si on applique la formule d'Euler-Rodrigues à un vecteur:
	
	\begin{eqnarray*}
	\bm{u}&=& \bm{R} \bm{v} \\
	&=& (\unit+sin(\phi)\bm{M}+(1-cos(\phi))\bm{M}^2) \bm{v} \\
	&=& \bm{v}+sin(\phi)\bm{M}\bm{v}+(1-cos(\phi))\bm{M}^2\bm{v}  \\
	&=& \bm{v}+sin(\phi)(\bm{n}\times \bm{v})+(1-cos(\phi))\bm{M} (\bm{n}\times \bm{v})  \\
	&=& \bm{v}+sin(\phi)(\bm{n}\times \bm{v})+(1-cos(\phi)) (\bm{n}\times (\bm{n}\times \bm{v}))  \\
	\end{eqnarray*}
	
Mais $\bm{u}\times(\bm{v}\times \bm{w})=(\bm{u}.\bm{w})\bm{v}-(\bm{u}.\bm{v})\bm{w}$ donc :
    
    \begin{eqnarray*}
    \bm{n}\times(\bm{n}\times \bm{v})&=&(\bm{n}.\bm{v})\bm{n}-(\bm{n}.\bm{n})\bm{v}\\
	&=&(\bm{n}.\bm{v})\bm{n}-\bm{v}\\	
	\end{eqnarray*}

Donc 
	\begin{eqnarray*}
		\bm{u}&=& \bm{v}+sin(\phi)(\bm{n}\times \bm{v})+(1-cos(\phi)) ((\bm{n}.\bm{v})\bm{n}-\bm{v})  \\
		&=& \bm{v}+sin(\phi)(\bm{n}\times \bm{v})+(1-cos(\phi)) ((\bm{n}.\bm{v})\bm{n})+cos(\phi)\bm{v}-\bm{v}  \\
	\end{eqnarray*}
Et finalement :
\[
\boxed{\bm{u}= cos(\phi)\bm{v}+sin(\phi)(\bm{n}\times \bm{v})+(1-cos(\phi)) (\bm{n}.\bm{v})\bm{n} }
\]

	\newpage
	\subsection{Axe et angle de rotation}
	
Soit $\bm{R}$ une matrice de rotation. Alors l'axe de rotation $\bm{n}$ est un vecteur propre de cette matrice de valeur propre 1 et aussi de la matrice de rotation inverse. Donc :

\[
\bm{R}\bm{n}=\bm{n} \qquad et \qquad \bm{R}^T\bm{n}=\bm{n}
\]  

Donc 

\[
\boxed{(\bm{R}-\bm{R}^T)\bm{n}=0}
\]

On peut alors remarquer que $(\bm{R}-\bm{R}^T)$ est une matrice anti-symétrique de trace nulle. Et si on pose 
$(\bm{R}-\bm{R}^T)=k \floor{n}_{\times}$ alors l'équation précédente est bien vérifiée car $k \bm{n}\times \bm{n}=0$. Pour identifier $k$ on peut calculer explicitement $(\bm{R}-\bm{R}^T)$ avec la formule d'Euler-Rodrigues ce qui donne:

\begin{eqnarray*}
	(\bm{R}-\bm{R}^T)&=& (\unit +\sin(\phi) \bm{M}+(1-\cos(\phi))\bm{M}^2)-(\unit^T +\sin(\phi) \bm{M}^T+(1-\cos(\phi))(\bm{M}^2)^T)\\
	&\qquad Mais\qquad& \bm{M}^T=-\bm{M}\\
	&\qquad Et \qquad& \bm{M}^2=(\bm{M}^2)^T\\
	&=& 2 \sin(\phi) \bm{M}\\
\end{eqnarray*}

Soit finalement:
\[
\boxed{(\bm{R}-\bm{R}^T)=2 \sin(\phi)\floor{n}_{\times}}
\]

Attention car il existe des matrices de rotation tel que $\bm{R}=\bm{R}^T$ qui correspondent au cas ou $sin(\phi)=0$, pour ces cas il est beaucoup plus stable numériquement de décomposer $\bm{R}$ en valeurs propres et de prendre le vecteur propre associée à la valeur propre de 1.\\

Maintenant nous avons aussi:

\[
	\boxed{tr(\bm{R})=1+2 \cos(\phi)}
\]

En effet :

\begin{eqnarray*}
tr(\bm{R})&=&tr(\unit)+(1-\cos(\phi)) tr(\bm{M}^2)\\
&=&3-(1-\cos(\phi))(n_z^2+n_y^2+n_z^2+n_x^2+n_y^2+n_x^2)\\
&=&3-(1-\cos(\phi))(2)\\
&=&1+2 \cos(\phi)\\
\end{eqnarray*}
	
	\newpage
	\subsection{Composition des rotations (en convention vecteurs colonnes)}
	
	Soit $\bm{A}$ et $\bm{B}$ deux repères orthonormés exprimés dans une base quelconque par des vecteurs colonnes:
	
	\[
		\bm{A}=\left(\begin{array}{c|c|c}
		&&\\
		\bm{a_x}&\bm{a_y}&\bm{a_z}\\
		&&\\
		\end{array}
		\right)		
		\qquad 
		et
		\qquad
		\bm{B}=\left(\begin{array}{c|c|c}
		&&\\
		\bm{b_x}&\bm{b_y}&\bm{b_z}\\
		&&\\
		\end{array}
		\right)
	\]
	
	
	Et soit $\bm{R}_{AB}$ la matrice de rotation telle que:
	
	\[
	\boxed{\bm{B}=\bm{A} \bm{R}_{ab}} \quad \quad donc \quad \quad \boxed{\bm{R}_{ab}=\bm{A}^T\bm{B}}
	\]
	
	Les rotations ainsi définies respectent la relation de Chasles. En effet si on introduit un troisième repère $\bm{C}$ et $\bm{R}_{ac}$,$\bm{R}_{cb}$ les matrices de rotation telles que:
	
	\[
	\bm{C}=\bm{A} \bm{R}_{ac}\quad et \quad \bm{B}=\bm{C} \bm{R}_{cb}  \quad\quad Soit\quad \quad \bm{R}_{ac}=\bm{A}^T\bm{C} \quad et\quad  \bm{R}_{cb}=\bm{C}^T\bm{B}
	\]
	
	Alors 
	
	\[
	\bm{R}_{ac}\bm{R}_{cb}=\bm{A}^T\bm{C} \bm{C}^T\bm{B} = \bm{A}^T\bm{B}
	\]
	Et donc 
	\[
	\boxed{\bm{R}_{ac}\bm{R}_{cb}=\bm{R}_{ab}}
	\]
	
	$\bm{R}_{ab}$ sont les vecteurs de la base $\bm{B}$ exprimés dans $\bm{A}$ c'est à dire que:
	
	\[
		\bm{R}_{ab}=\bm{A}^T\bm{B}=
		\left(\begin{array}{ccc}
		&\bm{a_x}&\\
		\hline
		&\bm{a_y}&\\
		\hline
		&\bm{a_z}&\\
		\end{array}
		\right)
		\left(\begin{array}{c|c|c}
		&&\\
		\bm{b_x}&\bm{b_y}&\bm{b_z}\\
		&&\\
		\end{array}
		\right)
		=
		\left(\begin{array}{ccc}
		\bm{a_x}.\bm{b_x}&\bm{a_x}.\bm{b_y}&\bm{a_x}.\bm{b_z}\\
		\bm{a_y}.\bm{b_x}&\bm{a_y}.\bm{b_y}&\bm{a_y}.\bm{b_z}\\
		\bm{a_z}.\bm{b_x}&\bm{a_z}.\bm{b_y}&\bm{a_z}.\bm{b_z}\\
		\end{array}
		\right)
	\]
	
	Maintenant, soit $\bm{v}_{a}$ et $\bm{v}_{b}$, des vecteurs exprimés respectivement dans les bases $\bm{A}$ et $\bm{B}$. Alors:
	
	\[
	\bm{A} \bm{v}_{a} =\bm{B} \bm{v}_{b} \quad\quad Soit\quad \quad \bm{v}_{a} =\bm{A}^T \bm{B} \bm{v}_{b}
	\]
	
	Par conséquent on a :
	
	\[
	\boxed{\bm{v}_{a} = \bm{R}_{ab} \bm{v}_{b}} 
	\]
	
	En résumé, pour tourner les repères on compose à droite. mais pour changer un vecteur de base on compose à gauche.
	
\newpage
\section{Représentation par $SU(2)$}
	\subsection{Les matrices de Pauli}
	 Introduisons les matrices $\sigma$ de Pauli:
	
	\[
	\bm{\sigma}_x=\begin{pmatrix}
	0&1\\
	1&0\\
	\end{pmatrix}\quad
	\bm{\sigma}_y=\begin{pmatrix}
	0&-i\\
	i&0\\
	\end{pmatrix}\quad
	\bm{\sigma}_z=\begin{pmatrix}
	1&0\\
	0&-1\\
	\end{pmatrix}\quad et\quad \bm{\sigma}=(\bm{\sigma}_x,\bm{\sigma}_y,\bm{\sigma}_z)^T
	\]
	
	On peut observer que le carré de ces matrices donne l'identité et que les matrices sont hermitiennes:  
	
	\[
		\boxed{\bm{\sigma}_i^2=\unit}\quad\quad et \quad\quad \boxed{\bm{\sigma}^{\dagger}=\bm{\sigma}}
	\]
	
	En effet:
	\[
	\bm{\sigma}_x^2=\begin{pmatrix}
	0&1\\
	1&0\\
	\end{pmatrix}
	\begin{pmatrix}
	0&1\\
	1&0\\
	\end{pmatrix}
	=
	\begin{pmatrix}
	1&0\\
	0&1\\
	\end{pmatrix}
	\]
	\[
	\bm{\sigma}_y^2=\begin{pmatrix}
	0&-i\\
	i&0\\
	\end{pmatrix}
	\begin{pmatrix}
	0&-i\\
	i&0\\
	\end{pmatrix}
	=
	\begin{pmatrix}
	1&0\\
	0&1\\
	\end{pmatrix}
	\]
	\[
	\bm{\sigma}_z^2=\begin{pmatrix}
	1&0\\
	0&-1\\
	\end{pmatrix}
	\begin{pmatrix}
	1&0\\
	0&-1\\
	\end{pmatrix}
	=
	\begin{pmatrix}
	1&0\\
	0&1\\
	\end{pmatrix}
	\]
	
	Et plus généralement l'anti-commutateur est nul si $i\ne j$:
	
	\[
		\boxed{ \{\bm{\sigma}_i,\bm{\sigma}_j\}=\bm{\sigma}_i\bm{\sigma}_j+\bm{\sigma}_j\bm{\sigma}_i=2 \delta_{ij} \unit}
	\]
	En effet:
	\[
	\bm{\sigma}_x\bm{\sigma}_y+\bm{\sigma}_y\bm{\sigma}_x=\begin{pmatrix}
	0&1\\
	1&0\\
	\end{pmatrix}
	\begin{pmatrix}
	0&-i\\
	i&0\\
	\end{pmatrix}
	+
	\begin{pmatrix}
	0&-i\\
	i&0\\
	\end{pmatrix}
	\begin{pmatrix}
	0&1\\
	1&0\\
	\end{pmatrix}	
	=
	\begin{pmatrix}
	i&0\\
	0&-i\\
	\end{pmatrix}
	+
	\begin{pmatrix}
	-i&0\\
	0&i\\
	\end{pmatrix}
	=0
	\]
	
	Mais pour les commutateurs on obtient:
	
	\[
	\boxed{[\bm{\sigma}_i,\bm{\sigma}_j]=2i \sum_k \epsilon_{ijk}\bm{\sigma}_k}
	\]
	En effet:
	\[
	\bm{\sigma}_x\bm{\sigma}_y-\bm{\sigma}_y\bm{\sigma}_x=\begin{pmatrix}
	0&1\\
	1&0\\
	\end{pmatrix}
	\begin{pmatrix}
	0&-i\\
	i&0\\
	\end{pmatrix}
	-
	\begin{pmatrix}
	0&-i\\
	i&0\\
	\end{pmatrix}
	\begin{pmatrix}
	0&1\\
	1&0\\
	\end{pmatrix}	
	=
	\begin{pmatrix}
	i&0\\
	0&-i\\
	\end{pmatrix}
	-
	\begin{pmatrix}
	-i&0\\
	0&i\\
	\end{pmatrix}
	=2i \bm{\sigma}_z
	\]
	
	Maintenant si on pose :
	
	\[
		\boxed{\bm{K}_i=-\frac{i}{2}\bm{\sigma}_i}
	\quad\quad alors \quad\quad
	[\bm{K}_i,\bm{K}_j]=-\frac{1}{4}[\bm{\sigma}_i,\bm{\sigma}_j]=-\frac{i}{2} \sum_k \epsilon_{ijk}\bm{\sigma}_k = \sum_k \epsilon_{ijk}\bm{K}_k
	\]
	
	Finalement les matrices $\bm{K}_i$ satisfont le crochet de Lie de $SO(3)$:
	\[
	\boxed{[\bm{K}_i,\bm{K}_j]= \sum_k \epsilon_{ijk}\bm{K}_k}
	\]
	
\newpage
\subsection{Calcul de l'exponentielle pour $SU(2)$}

Tout groupe de 3 objets vérifiant l'algèbre de lie de $SO(3)$ fournit une représentation du groupe de rotations. Donc les matrices $K_i$ fournissent une représentation du groupe de rotation par l'application exponentielle:

\[
\boxed{\bm{U}(\bm{n},\phi) = e^{\bm{n}.\bm{K} \phi}= e^{-\frac{i}{2}\bm{n}.\bm{\sigma} \phi}}
\]

On remarque alors que les rotations sont représentées p par une matrice unitaire $\bm{U}$, c'est à dire tel que:

\[
\boxed{\bm{U}^{\dagger}\bm{U}=\unit}
\] 

En effet:

\begin{eqnarray*}
\bm{U}^{\dagger}\bm{U}&=&\left(e^{-\frac{i}{2}\bm{n}.\bm{\sigma} \phi}\right)^\dagger e^{-\frac{i}{2}\bm{n}.\bm{\sigma} \phi}\\
&=&e^{\frac{i}{2}\bm{n}.\bm{\bm{\sigma}^\dagger} \phi} e^{-\frac{i}{2}\bm{n}.\bm{\sigma} \phi}\\
&=&e^{\frac{i}{2}\bm{n}.\bm{\sigma} \phi} e^{-\frac{i}{2}\bm{n}.\bm{\sigma} \phi}\\
&=&\unit\\
\end{eqnarray*}

De manière analogue à ce que nous avions fait pour établir la formule de Rodrigue. posons :

\[
	\bm{M}=\bm{n}.\bm{\sigma}=(n_x \bm{\sigma}_x+n_y \bm{\sigma}_y+n_z \bm{\sigma}_z)= \begin{pmatrix}
	n_z&n_x-i n_y\\
	n_x+i n_y&-n_z\\
	\end{pmatrix}
\]

Calculons $\bm{M}^2$:

\[
	\bm{M}^2=\begin{pmatrix}
	n_z&n_x-i n_y\\
	n_x+i n_y&-n_z\\
	\end{pmatrix}
	\begin{pmatrix}
	n_z&n_x-i n_y\\
	n_x+i n_y&-n_z\\
	\end{pmatrix}
	=
	\begin{pmatrix}
	n_x^2+ n_y^2+n_z^2&0\\
	0&n_x^2+ n_y^2+n_z^2\\
	\end{pmatrix}
\]

Donc comme $|\bm{n}|=1$ on obtient que $\boxed{\bm{M}^2=\unit}$. Calculons maintenant l'exponentielle:

\begin{eqnarray*}
\bm{U}(\bm{n},\phi) &=& e^{-\frac{i}{2} \bm{M} \phi}\\
&=& \sum_{n=0}^{\infty} \frac{(-i \bm{M} \frac{\phi}{2})^n}{n!} \\
&=& \sum_{n=0}^{\infty} \frac{(-i \bm{M} \frac{\phi}{2})^{2n}}{(2n)!} +\sum_{n=0}^{\infty} \frac{(-i \bm{M} \frac{\phi}{2})^{2n+1}}{(2n+1)!} \\
&=& \unit \sum_{n=0}^{\infty} \frac{(-1)^n (\frac{\phi}{2})^{2n}}{(2n)!} +\bm{M}\sum_{n=0}^{\infty} \frac{(-i)^{2n+1} (\frac{\phi}{2})^{2n+1}}{(2n+1)!} \\
&=& \unit \sum_{n=0}^{\infty} \frac{(-1)^n (\frac{\phi}{2})^{2n}}{(2n)!}-i \bm{M}\sum_{n=0}^{\infty} \frac{(-1)^{n} (\frac{\phi}{2})^{2n+1}}{(2n+1)!} \\
&=& \unit \cos\left(\frac{\phi}{2}\right)-i \bm{M} \sin\left(\frac{\phi}{2}\right)\\
\end{eqnarray*}

Finalement :

\[
\boxed{\bm{U}(\bm{n},\phi)= \unit \cos\left(\frac{\phi}{2}\right)-i \bm{n}.\bm{\sigma} \sin\left(\frac{\phi}{2}\right)}
\]

\newpage
\subsection{Interpolation Sphérique : Slerp}

Si on part de l'expression de l'exponentielle sur $SU(2)$: 
\[
\bm{U}(\bm{n},\phi)=e^{-\frac{i}{2} \bm{M} \phi}= \unit \cos\left(\frac{\phi}{2}\right)-i \bm{n}.\bm{\sigma} \sin\left(\frac{\phi}{2}\right)
\]
On peut constater que élever la matrice à une puissance non entière $\alpha$ donne:
\[
\bm{U}^\alpha=\left(e^{-\frac{i}{2} \bm{M} \phi}\right)^\alpha=e^{-\frac{i}{2}\alpha \bm{M} \phi}
\]
Finalement on trouve:
\[
\boxed{\bm{U}^\alpha=\unit \cos\left(\frac{\phi\alpha}{2}\right)-i \bm{n}.\bm{\sigma} \sin\left(\frac{\phi\alpha}{2}\right)}
\]
On peut donc interpoler entre deux rotations avec :

\[
\boxed{Slerp(\bm{U}_a,\bm{U}_b,\alpha)= \bm{U_a}\left(\bm{U_a}^{-1} \bm{U_b}\right)^{\alpha} \quad avec\quad \alpha\in[0,1]}
\]


\newpage
\subsection{Action de $SU(2)$}

A chaque vecteur $\bm{r}$ de $\mathbb{R}^3$, on peut faire correspondre la matrice $\bm{X} \in SU(2)$ tel que:

\[
	\boxed{\bm{X}=\bm{r}.\bm{\sigma}} \quad soit \quad \bm{X}=\begin{pmatrix}
	r_z&r_x-i r_y\\
	r_x+i r_y&-r_z\\
	\end{pmatrix}
\]

Calculons le produit de deux telles matrices:
\begin{eqnarray*}
(\bm{a}.\bm{\sigma}) (\bm{b}.\bm{\sigma})&=&\begin{pmatrix}
a_z&a_x-i a_y\\
a_x+i a_y&-a_z\\
\end{pmatrix}
\begin{pmatrix}
b_z&b_x-i b_y\\
b_x+i b_y&-b_z\\
\end{pmatrix}\\
&=&
\begin{pmatrix}
(b_x a_x+ b_y a_y+a_z b_z)  +i (b_y a_x- b_x a_y)& (b_x a_z- b_z a_x) + i (b_z a_y- b_y a_z)  \\
(b_z a_x  - b_x a_z) + i(b_z a_y-b_y a_z)& (b_x a_x +b_y a_y+ b_z a_z)+ i (b_x a_y - b_y a_x )\\
\end{pmatrix}\\
&=&
\unit \bm{a}.\bm{b}+ i\begin{pmatrix}
(b_y a_x- b_x a_y)&   (b_z a_y- b_y a_z)-i(b_x a_z- b_z a_x)  \\
 (b_z a_y-b_y a_z)+i(b_x a_z - b_z a_x)& (b_x a_y - b_y a_x )\\
\end{pmatrix}
\end{eqnarray*}

Et si on pose :

\[
\bm{c}=\begin{pmatrix}
b_z a_y- b_y a_z\\
b_x a_z- b_z a_x\\
b_y a_x- b_x a_y\\
\end{pmatrix}= \bm{a}\times\bm{b}
\]

Donc finalement on a la relation :

\[
 \boxed{ (\bm{a}.\bm{\sigma}) (\bm{b}.\bm{\sigma}) = \bm{a}.\bm{b} \unit + i(\bm{a}\times\bm{b}).\bm{\sigma}}
\]

On peut inverser la carte par la relation:

\[
\boxed{r_i= \frac{1}{2} Tr(\bm{X}.\sigma_i)}
\]

On peut alors agir sur $\bm{X}$ par :

\[
	\boxed{ \bm{X}'=\bm{U} \bm{X} \bm{U}^{\dagger} }
\]

\subsection{Revêtement double de $SO(3)$ et topologie}
\label{l_topo_su2}

On peut constater trivialement que $\bm{U}$ et $-\bm{U}$ engendre la même action.
En ce sens $SU(2)$ constitue un revêtement double de $SO(3)$.\\

Toute matrice unitaire peut se mettre sous la forme:

\[
\bm{U}=
\begin{pmatrix}
\alpha & \beta\\
-\beta^* & \alpha^*\\
\end{pmatrix}
=
\begin{pmatrix}
a+b i & c+d i\\
-c+d i & a-b i\\
\end{pmatrix}
\quad 
avec 
\quad 
|\alpha|^2+|\beta|^2=1 \quad \rightarrow \quad a^2+b^2+c^2+d^2=1
\]

Topologiquement $SU(2)$ est donc isomorphe à $S^3$, la sphere de $\mathbb{R}^4$. 


\newpage
\subsection{Euler-Rodrigue par l'action de $SU(2)$}

En partant de l'action de $SU(2)$ on peut redémontrer la formule d'Euler-Rodrigue. Ce qui donne en s'armant de courage:

\begin{eqnarray*}
	\bm{r'}.\bm{\sigma}&=&\bm{U} (\bm{r}.\bm{\sigma}) \bm{U}^{\dagger}\\
	&=&\left(\unit \cos\left(\frac{\phi}{2}\right)-i \bm{n}.\bm{\sigma} \sin\left(\frac{\phi}{2}\right)\right) (\bm{r}.\bm{\sigma}) \left(\unit \cos\left(\frac{\phi}{2}\right)-i \bm{n}.\bm{\sigma} \sin\left(\frac{\phi}{2}\right)\right)^{\dagger}\\
	&=&\left(\unit \cos\left(\frac{\phi}{2}\right)-i \bm{n}.\bm{\sigma} \sin\left(\frac{\phi}{2}\right)\right) (\bm{r}.\bm{\sigma}) \left(\unit \cos\left(\frac{\phi}{2}\right)+i \bm{n}.\bm{\sigma} \sin\left(\frac{\phi}{2}\right)\right)\\
	&=&\left((\bm{r}.\bm{\sigma}) \cos\left(\frac{\phi}{2}\right)-i (\bm{n}.\bm{\sigma})(\bm{r}.\bm{\sigma}) \sin\left(\frac{\phi}{2}\right)\right)  \left(\unit \cos\left(\frac{\phi}{2}\right)+i \bm{n}.\bm{\sigma} \sin\left(\frac{\phi}{2}\right)\right)\\
	&=&\left((\bm{r}.\bm{\sigma}) \cos^2\left(\frac{\phi}{2}\right)-i (\bm{n}.\bm{\sigma})(\bm{r}.\bm{\sigma}) \sin\left(\frac{\phi}{2}\right)\cos\left(\frac{\phi}{2}\right)\right) + \\& & \left(i(\bm{r}.\bm{\sigma})(\bm{n}.\bm{\sigma}) \cos\left(\frac{\phi}{2}\right)\sin\left(\frac{\phi}{2}\right)+ (\bm{n}.\bm{\sigma})(\bm{r}.\bm{\sigma})(\bm{n}.\bm{\sigma}) \sin^2\left(\frac{\phi}{2}\right)\right)\\
	&=&\left((\bm{r}.\bm{\sigma}) \cos^2\left(\frac{\phi}{2}\right)-i (\bm{n}.\bm{r} \unit + i(\bm{n}\times\bm{r}).\bm{\sigma}) \sin\left(\frac{\phi}{2}\right)\cos\left(\frac{\phi}{2}\right)\right) + \\& & \left(i(\bm{r}.\bm{n} \unit + i(\bm{r}\times\bm{n}).\bm{\sigma}) \cos\left(\frac{\phi}{2}\right)\sin\left(\frac{\phi}{2}\right)+ (\bm{n}.\bm{\sigma})(\bm{r}.\bm{\sigma})(\bm{n}.\bm{\sigma}) \sin^2\left(\frac{\phi}{2}\right)\right)\\
	&=&(\bm{r}.\bm{\sigma}) \cos^2\left(\frac{\phi}{2}\right)-i (\bm{n}.\bm{r} \unit + i(\bm{n}\times\bm{r}).\bm{\sigma}-\bm{r}.\bm{n} \unit - i(\bm{r}\times\bm{n}).\bm{\sigma}) \sin\left(\frac{\phi}{2}\right)\cos\left(\frac{\phi}{2}\right) + \\& & (\bm{n}.\bm{\sigma})(\bm{r}.\bm{\sigma})(\bm{n}.\bm{\sigma}) \sin^2\left(\frac{\phi}{2}\right)\\
	&=&\frac{1}{2}(\bm{r}.\bm{\sigma}) (\cos(\phi)+1)+ 2((\bm{n}\times\bm{r}).\bm{\sigma}) \sin\left(\frac{\phi}{2}\right)\cos\left(\frac{\phi}{2}\right) + (\bm{n}.\bm{\sigma})(\bm{r}.\bm{n} \unit + i(\bm{r}\times\bm{n}).\bm{\sigma}) \sin^2\left(\frac{\phi}{2}\right)\\
	&=&\frac{1}{2}(\bm{r}.\bm{\sigma}) (\cos(\phi)+1)+ 2((\bm{n}\times\bm{r}).\bm{\sigma}) \sin\left(\frac{\phi}{2}\right)\cos\left(\frac{\phi}{2}\right) + (\bm{r}.\bm{n})(\bm{n}.\bm{\sigma})\sin^2\left(\frac{\phi}{2}\right)+ i(\bm{n}.\bm{\sigma})((\bm{r}\times\bm{n}).\bm{\sigma}) \sin^2\left(\frac{\phi}{2}\right)\\
	&=&\left(\frac{1}{2} (\cos(\phi)+1) \bm{r} + \sin(\phi)(\bm{n}\times\bm{r}) + \sin^2\left(\frac{\phi}{2}\right)(\bm{r}.\bm{n})\bm{n}\right).\bm{\sigma}+ i(\bm{n}.\bm{\sigma})((\bm{r}\times\bm{n}).\bm{\sigma}) \sin^2\left(\frac{\phi}{2}\right)\\
	&=&\left(\frac{1}{2} (\cos(\phi)+1) \bm{r} + \sin(\phi)(\bm{n}\times\bm{r}) + \sin^2\left(\frac{\phi}{2}\right)(\bm{r}.\bm{n})\bm{n}\right).\bm{\sigma}+ i( \bm{n}.(\bm{r}\times \bm{n}) \unit + i(\bm{n}\times (\bm{r}\times \bm{n})).\bm{\sigma} ) \sin^2\left(\frac{\phi}{2}\right)\\
	&=&\left(\frac{1}{2} (\cos(\phi)+1) \bm{r} + \sin(\phi)(\bm{n}\times\bm{r}) + \sin^2\left(\frac{\phi}{2}\right)(\bm{r}.\bm{n})\bm{n}\right).\bm{\sigma}-( ((\bm{n}.\bm{n})\bm{r} - (\bm{n}.\bm{r})\bm{n}).\bm{\sigma} )	
	\sin^2\left(\frac{\phi}{2}\right)\\	
	&=&\left(\frac{1}{2} (\cos(\phi)+1) \bm{r} + \sin(\phi)(\bm{n}\times\bm{r}) + \frac{1}{2}(1-\cos(\phi)) (-\bm{r} + 2 (\bm{n}.\bm{r})\bm{n} ) \right).\bm{\sigma}\\	
	&=&\left(\cos(\phi) \bm{r}+\frac{1}{2}(1-\cos(\phi))\bm{r} + \sin(\phi)(\bm{n}\times\bm{r}) + \frac{1}{2}(1-\cos(\phi)) (-\bm{r} + 2 (\bm{n}.\bm{r})\bm{n}  ) \right).\bm{\sigma}\\
	&=&\left(\cos(\phi) \bm{r} + \sin(\phi)(\bm{n}\times\bm{r}) + (1-\cos(\phi)) ( (\bm{n}.\bm{r})\bm{n}  ) \right).\bm{\sigma}\\
\end{eqnarray*}

Ce qui redonne bien la formule d'Euler-Rodrigue vectorielle:

\[
\boxed{\bm{r}'=\cos(\phi) \bm{r} + \sin(\phi)(\bm{n}\times\bm{r}) + (1-\cos(\phi)) ( (\bm{n}.\bm{r})\bm{n}  )}
\]

\subsection{Isomorphisme avec les quaternions unitaires Sp(1)}

On a l'isomorphisme avec les quaternions unitaires $SU(2) \overset{iso.}{\longrightarrow} \{\mathbb{H},|.|=1\}=Sp(1)$. En effet, on a le mapping:
\[
	\boxed{\bm{U}=
	\begin{pmatrix}
	a+b i & c+d i\\
	-c+d i & a-b i\\
	\end{pmatrix}=a \unit+ b i\bm{\sigma}_z + c i \bm{\sigma}_y + d i \bm{\sigma}_x
	\qquad
	Donne\ le\ quaternion
	\qquad
	Q=a+b i+c j+d k}
\]

Comme $det(\bm{U})=1=a^2+b^2+c^2+d^2$ le quaternion est unitaire (ie $|Q|=1$). Et finalement les unités $i,j,k$ peuvent s'exprimer par leurs matrices équivalentes $\bm{I},\bm{J},\bm{K} \in SU(2)$:

\[
	\boxed{\bm{I}=i\sigma_z \quad \bm{J}=i\sigma_y \quad \bm{K}=i\sigma_x}
\]  

Soit clairement

\[
	\boxed{\bm{U}=a\mathds{1}+b\bm{I}+c\bm{J}+d\bm{K}}
\]

Les quaternions unitaires et $SU(2)$ ont donc la même topologie \ref{l_topo_su2}.

\newpage
\section{Annexes}
\subsection{Retrouver l'identité à partir de la formule d'Euler-Rodrigue}
Avec $\bm{M}^T=-\bm{M}$ et $(\bm{M}^2)^T=\bm{M}^2$ et $\bm{M}^3=-\bm{M}$ on verifie que:

\begin{eqnarray*}
	\bm{R}^T \bm{R}&=&(\unit+sin(\phi)\bm{M}+(1-cos(\phi))\bm{M}^2)^T (\unit+sin(\phi)\bm{M}+(1-cos(\phi))\bm{M}^2)\\
	&=&\unit+sin(\phi)\bm{M}^T+(1-cos(\phi))(\bm{M}^2)^T+\\
	&&(\unit+sin(\phi)\bm{M}^T+(1-cos(\phi))(\bm{M}^2)^T)sin(\phi)\bm{M}+\\
	&&(\unit+sin(\phi)\bm{M}^T+(1-cos(\phi))(\bm{M}^2)^T)(1-cos(\phi))\bm{M}^2\\
	&=&\unit+sin(\phi)\bm{M}^T+(1-cos(\phi))(\bm{M}^2)^T+\\
	&&sin(\phi)\bm{M}+sin(\phi)sin(\phi)\bm{M}^T\bm{M}+(1-cos(\phi))sin(\phi)(\bm{M}^2)^T\bm{M}+\\
	&&(1-cos(\phi))\bm{M}^2+sin(\phi)(1-cos(\phi))\bm{M}^T\bm{M}^2+(1-cos(\phi))^2(\bm{M}^2)^T\bm{M}^2\\
	&=&\unit-sin(\phi)\bm{M}+(1-cos(\phi))\bm{M}^2+\\
	&&sin(\phi)\bm{M}-sin(\phi)sin(\phi)\bm{M}^2-(1-cos(\phi))sin(\phi)\bm{M}+\\
	&&(1-cos(\phi))\bm{M}^2-sin(\phi)(1-cos(\phi))\bm{M}^3-(1-cos(\phi))^2\bm{M}^2\\
	&=&\unit+(2(1-cos(\phi))-sin(\phi)^2-(1-cos(\phi))^2)\bm{M}^2\\
	&=&\unit+(2-2cos(\phi)-sin(\phi)^2-1+2cos(\phi)-cos(\phi)^2)\bm{M}^2\\
	&=&\unit+(1-sin(\phi)^2-cos(\phi)^2)\bm{M}^2\\
	&=&\unit\\	
\end{eqnarray*}

\subsection{Distances entre rotations}
	\begin{tabular}{|l|l|c|}
		\hline
		Nom& Definition& Range\\
		\hline
		\hline
		Distance euclidienne entre 2 quaternions & $D_1(Q_a,Q_b)=min(||Q_a-Q_b||,||Q_a+Q_b||])$& $[0,\sqrt{2}]$ \\
		Produit scalaire entre 2 quaternions & $D_2(Q_a,Q_b)=acos(|Q_a.Q_b|)$& $[0,\frac{\pi}{2}]$ \\
		Produit scalaire entre 2 quaternions & $D_3(Q_a,Q_b)=1-|Q_a.Q_b|$& $[0,1]$ \\
		Déviation de l'identité &  $D_4(R_a,R_b)=||\unit-R_a.R_b^T||_F$& $[0,2\sqrt{2}]$ \\		
		 &  $D_4(Q_a,Q_b)=2\sqrt{2 D_3(Q_a,Q_b)}$& $[0,2\sqrt{2}]$ \\
		Distance euclidienne sur $R^9$ &  $D_5(R_a,R_b)=||R_a-R_b||_F=D_4$& $[0,2\sqrt{2}]$  \\
		Géodésique of $S^3$ &  $D_5(R_a,R_b)=||log(R_a R_b^T)||_F=angle(R_a,R_b)$& $[0,\pi]$ \\
		\hline
	\end{tabular}

\newpage
\subsection{Trouver la rotation la plus proche}
\label{l_rot_plus_proche_d5}

Soit $\hat{R}$ une estimation d'une matrice de rotation. Si on décompose en valeurs singulières $\hat{R}=U \Sigma V^*$, alors:

\[
	\boxed{R=UV^*}\qquad alors\ on\ a \qquad \boxed{R=argmin ||R-\hat{R}||_F}
\]
	
$R$ est la matrice de rotation la plus proche de $\hat{R}$ au sens de la distance $D_5$.

\subsection{Rotation moyenne SO(3)}

Soit $R_k$ une collection de $N$ rotations, on souhaite trouver $R$ une rotation moyenne au sens ou la somme suivante est minimale:

\[
	\boxed{C=\sum_k^N ||R_k-R||_F^2}
\]

On calcul d'abord :

\[
	\hat{R}=\frac{1}{N}\sum_k^N R_k
\]

Puis par on cherche la rotation la plus proche de $\hat{R}$ avec la méthode \ref{l_rot_plus_proche_d5}. Si on décompose en valeurs singulières $\hat{R}=U \Sigma V^*$

\[
\boxed{R=UV^*}\qquad alors\ on\ a \qquad \boxed{R=argmin ||R-\hat{R}||_F}
\]

Alors $R$ est l'unique rotation minimisant la somme des écarts au carré au sens de la distance $D_5$.

\subsection{Rotation moyenne pondérée Sp(1)}

Soit $N$ quaternions $\bm{q}_i$ exprimés commre vecteurs de $\mathbb{R}^4$ et $N$ poids $\omega_i$ On construit la matrice 4x4 .\\

\[
\bm{M}=\sum_{n=1}^N \omega_i \bm{q}_i \bm{q}^T_i
\]

Alors le quaternion moyen est donné par 

\[
\bm{\hat{q}}= \argmax_{q\in S^3} \bm{q} \bm{M} \bm{q}^T
\]

La solution est alors le vecteur propre de $\bm{M}$ associé à la plus grande valeur propre de $\bm{M}$. On veillera à ce que le vecteur propre soit bien de norme unitaire. Notons que l'ambiguïté du signe est bien prise en compte dans la construction de $\bm{M}$.

\end{document}