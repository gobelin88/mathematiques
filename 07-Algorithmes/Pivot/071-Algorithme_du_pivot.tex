

\documentclass[12pt,a4paper]{article}
\usepackage{geometry}
\usepackage{amsmath}
\usepackage{amsfonts}
\usepackage{graphicx}
\geometry{hmargin=1.0cm,vmargin=2.0cm}
\setlength{\parindent}{0cm}
\title{Algorithme du pivot}
\author{Loïc Huguel}

\usepackage{mathtools}
\DeclarePairedDelimiter\ceil{\lceil}{\rceil}
\DeclarePairedDelimiter\floor{\lfloor}{\rfloor}


\begin{document}
	\maketitle
	
	\tableofcontents
	
	\newpage
	\section{Le point pivot}
	
	Soit deux référentiels $\mathbf{A}$ et $\mathbf{B}$ et un point $\mathbf{V}$, on a :
	
	\[
		\mathbf{B}=\mathbf{A} \mathbf{R}_{AB} +\mathbf{T}_{AB} \qquad et \qquad
		\mathbf{V}_A= \mathbf{R}_{AB} \mathbf{V}_B +\mathbf{T}_{AB}
	\]
	
	Si on a une collection de $n$ relations acquises par un point pivot sur $V$ dans le référentiel $A$, alors:	
	
	\[
	\left\{
	\begin{array}{ccc}
	\mathbf{V}_A&=& \mathbf{R}^1_{AB} \mathbf{V}_B +\mathbf{T}^1_{AB}\\
	\mathbf{V}_A&=& \mathbf{R}^2_{AB} \mathbf{V}_B +\mathbf{T}^2_{AB}\\
	...&=&...\\
	\mathbf{V}_A&=& \mathbf{R}^n_{AB} \mathbf{V}_B +\mathbf{T}^n_{AB}\\
	\end{array}
	\right.
	\qquad
	Soit
	\qquad
	\left(
	\begin{array}{c|c}
	\mathbf{I} & -\mathbf{R}^1_{AB}\\
	\mathbf{I} & -\mathbf{R}^2_{AB}\\
	...&...\\
	\mathbf{I} & -\mathbf{R}^n_{AB}\\
	\end{array}
	\right)		
	\left(
	\begin{array}{c}
	\mathbf{V}_A \\
	\mathbf{V}_B \\
	\end{array}
	\right)
	=
	\left(
	\begin{array}{c}
	\mathbf{T}^1_{AB} \\
	\hline
	\mathbf{T}^2_{AB} \\
	\hline
	...\\
	\hline
	\mathbf{T}^n_{AB}
	\end{array}
	\right)
	\]

	Si on pose 
	
	\[
		\mathbf{A}=\left(
		\begin{array}{c|c}
		\mathbf{I} & -\mathbf{R}^1_{AB}\\
		\mathbf{I} & -\mathbf{R}^2_{AB}\\
		...&...\\
		\mathbf{I} & -\mathbf{R}^n_{AB}\\
		\end{array}
		\right)
		\qquad
		\mathbf{X}=\left(
		\begin{array}{c}
		\mathbf{V}_A \\
		\mathbf{V}_B \\
		\end{array}
		\right)
		\qquad
		et
		\qquad
		\mathbf{B}=
		\left(
		\begin{array}{c}
		\mathbf{T}^1_{AB} \\
		\hline
		\mathbf{T}^2_{AB} \\
		\hline
		...\\
		\hline
		\mathbf{T}^n_{AB}
		\end{array}
		\right)
	\]
	
	On peut alors minimiser $\mathbf{A}\mathbf{X}=\mathbf{B}$ par les équations normales:
	
	\[
		\boxed{\hat{\mathbf{X}}=(\mathbf{A}^T\mathbf{A})^{-1} \mathbf{A}^T \mathbf{B}}
	\]
	
	L'erreur minimisée est alors $|\mathbf{A}\mathbf{X}-\mathbf{B}|^2$ soit:
	
	\[
		\boxed{E=\sum_{i=0}^n |\mathbf{V}_A - \mathbf{R}_{AB}^i \mathbf{V}_B +\mathbf{T}_{AB}^i|^2}
	\]
	
	On obtient alors deux résultats qui sont les coordonnées du point $\mathbf{V}$ dans les référentiels $\mathbf{A}$ et $\mathbf{B}$ respectivement $\mathbf{V}_A$ et $\mathbf{V}_B$. Tel que la distance entre les deux points estimés soit minimale pour chaque transformation.\\
	
	\textbf{Avantages/Inconvénients:}
	\begin{itemize}
		\item $-$ Attention à ne pas rester sur place.
		\item $-$ Si on bouge peu, la méthode est instable.	
		\item $+$ Chaque point compte autant qu'un autre indépendamment de la vitesse et/ou fréquence d'acquisition.
		\item $+$ Estimation de $\mathbf{V}_A$ et $\mathbf{V}_B$.			
	\end{itemize}
	
	\newpage
	\section{Variante dit LMP "Least Moving Point"}
	
	On peut éliminer $\mathbf{V}_A$ en effectuant les différences 2 à deux des équations du point pivot, cela donne:
	
	\[
	\left\{
	\begin{array}{ccc}
	0&=& (\mathbf{R}^1_{AB}-\mathbf{R}^2_{AB}) \mathbf{V}_B +\mathbf{T}^1_{AB}-\mathbf{T}^2_{AB}\\
	0&=& (\mathbf{R}^2_{AB}-\mathbf{R}^3_{AB}) \mathbf{V}_B +\mathbf{T}^2_{AB}-\mathbf{T}^3_{AB}\\
	...&=&...\\
	0&=& (\mathbf{R}^{n-1}_{AB}-\mathbf{R}^n_{AB}) \mathbf{V}_B +\mathbf{T}^{n-1}_{AB}-\mathbf{T}^n_{AB}\\
	\end{array}
	\right.
	\qquad
	Posons
	\qquad
	\left\{
	\begin{array}{l} 
	\mathbf{R^n}=\mathbf{R}^{n}_{AB}-\mathbf{R}^{n+1}_{AB}\\
	\mathbf{T^n}=\mathbf{T}^{n}_{AB}-\mathbf{T}^{n+1}_{AB}\\
	\end{array}
	\right.
	\]
	
	Sous forme matricielle on obtient alors
	\[
	\qquad
	\left(
	\begin{array}{c}
	-\mathbf{R}^1\\
	-\mathbf{R}^2\\
	...\\
	-\mathbf{R}^{n-1}\\
	\end{array}
	\right)		
	\left(
	\begin{array}{c}
	\mathbf{V}_B \\
	\end{array}
	\right)
	=
	\left(
	\begin{array}{c}
	\mathbf{T}^1 \\
	\hline
	\mathbf{T}^2 \\
	\hline
	...\\
	\hline
	\mathbf{T}^{n-1}
	\end{array}
	\right)
	\]
	
	Maintenant si on pose 
	
	\[
	\mathbf{A}=\left(
	\begin{array}{c}
	-\mathbf{R}^1\\
	-\mathbf{R}^2\\
	...\\
	-\mathbf{R}^{n-1}\\
	\end{array}
	\right)		
	\qquad 
	\mathbf{X}=\mathbf{V}_B
	\qquad
	et
	\qquad
	\mathbf{B}=\left(
	\begin{array}{c}
	\mathbf{T}^1 \\
	\hline
	\mathbf{T}^2 \\
	\hline
	...\\
	\hline
	\mathbf{T}^{n-1}
	\end{array}
	\right)	
	\]
	
	On peut alors minimiser $\mathbf{A}\mathbf{X}=\mathbf{B}$ par les équations normales et l'erreur minimisée est alors $E=|\mathbf{A}\mathbf{X}-\mathbf{B}|^2$ soit:
	
	\[
	\boxed{\hat{\mathbf{X}}=(\mathbf{A}^T\mathbf{A})^{-1} \mathbf{A}^T \mathbf{B}}
	\qquad et \qquad \boxed{E=\sum_{i=0}^{n-1} |\mathbf{R}^i \mathbf{V}_B+\mathbf{T}^i|^2}
	\]
	
	Cette erreur correspond à la somme des distances parcourues par le point $\mathbf{V}_A$. En effet 
	
	\begin{eqnarray*}
		d_A(\mathbf{V}_B) &=& \sum_i^{n-1} |\mathbf{V}_A^{i}-\mathbf{V}_A^{i+1}|^2\\
		&=&\sum_i^{n-1} | \mathbf{R}_{AB}^{i} \mathbf{V}_B +\mathbf{T}^{i}_{AB}-(\mathbf{R}_{AB}^{i+1} \mathbf{V}_B +\mathbf{T}^{i+1}_{AB})|^2 \\
		&=&\sum_i^{n-1} | (\mathbf{R}_{AB}^{i}-\mathbf{R}_{AB}^{i+1}) \mathbf{V}_B +\mathbf{T}^{i}_{AB}- \mathbf{T}^{i+1}_{AB}|^2 \\
		&=&\sum_i^{n-1} | \mathbf{R}^{i} \mathbf{V}_B +\mathbf{T}^{i}|^2 \\
	\end{eqnarray*}	
	
	Cet algorithme trouve donc le point $\mathbf{V}_B$ qui minimise $d_A(\mathbf{V}_B)$.\\
	
	
	\textbf{Avantages/Inconvénients:}
	\begin{itemize}
		\item $-$ Dépendant de la vitesse de mouvement et/ou de la fréquence d'acquisition.
		\item $+$ Si on reste sur place c'est pas grave.
		\item $-$ Si on bouge peu, la méthode est instable.
		\item $\pm$ La méthode peu être améliorer en considérant tous les couples possible d'équations pour substituer $\mathbf{V}_A$ ce qui ferait $n^2$ équations.
	\end{itemize} 

\end{document}